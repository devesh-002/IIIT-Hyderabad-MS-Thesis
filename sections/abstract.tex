\begin{sloppypar}
    This thesis investigates the influence of geography on politics and social structures in India. This study specifically views geography as a driving force that actively shapes how  politics evolve by particularly examining deviations from Duverger’s Law. According to Duverger’s Law, single\hyp{}member plurality electoral systems typically result in two\hyp{}party dominance. However, India is  a notable exception to this law. 
\end{sloppypar}
\begin{sloppypar}
    
To test the hypothesis we perform a quantitative analysis on electoral data comparing India’s Himalayan states to those in the Gangetic and Brahmaputra plains. The electoral data is operationalized using the  Effective Number of Parties (ENP) which measures the degree of political fragmentation across parliamentary and assembly constituencies in both mountain and plain states. The results indicate that the plains are diverging from while the mountain states are converging towards Duverger's two-party equilibrium. Complementing the electoral data, this study also explores the gender dynamics of mountain and plain societies by using data from the National Family Health Survey and Census data over various parameters -- namely, literacy rates, child marriage prevalence, contraceptive use, and breastfeeding practices. The parameters indicate the personal liberties of women and their agency in making decisions that affect their personal and family lives. Upon aggregating these parameters into a composite ranking system, we find that women in mountain states generally have more autonomy than those in plain states. To explain this, this thesis draws on anthropological and historical scholarship which postulates that mountain societies are structurally different from plain societies with regard to political and gender dynamics. Revisiting the theories of identity politics and strategic voting suggests that strategic voting alone cannot account for the observed deviations from Duverger's Law. This study engages with the idea of Zomia, first presented by Schendel and elaborated by Scott, which hypothesizes that highland communities across Asia are structurally different from plain communities, and have historically resisted state incorporation; developing more egalitarian and decentralized social systems. The results observed in our quantitative analyses are consistent with Scott's hypothesis which suggests that geographic differences \textit{might} be a reason for the difference. The study also includes detailed case studies of Himachal Pradesh and Manipur to illustrate how these dynamics manifest within individual states. By integrating electoral theory, sociological data, and regional case studies, this thesis offers geographical differences as a potential parameter to the understanding of Indian democracy. 
\end{sloppypar}