This thesis investigates the influence of geography on political and social structures in India, particularly examining deviations from Duverger’s Law. According to Duverger’s Law, single-member plurality electoral systems typically result in two-party dominance; however, India represents a notable exception. This research hypothesizes that India's deviation from Duverger’s Law is predominantly influenced by electoral patterns from densely populated regions, especially the Indo-Gangetic plains.

To evaluate this hypothesis, the thesis employs quantitative analysis of electoral data, comparing India's Himalayan states with those in the Gangetic and Brahmaputra plains. The Effective Number of Parties (ENP) is used as a measure to assess political fragmentation across parliamentary and assembly constituencies in both mountainous and plain regions. Results indicate that plain regions exhibit increasing political fragmentation, whereas mountainous states appear to be converging toward the two-party equilibrium predicted by Duverger’s Law.

Complementing electoral analysis, the study explores gender dynamics using data from the National Family Health Survey and Census, focusing on indicators such as literacy rates, child marriage prevalence, contraceptive use, and breastfeeding practices. These indicators are aggregated into a composite ranking system, revealing greater female autonomy in mountainous regions compared to plain regions. To contextualize these findings, the thesis draws upon anthropological and historical scholarship, emphasizing distinct socio-political behaviors in mountainous societies.

Critically assessing theories of identity politics and strategic voting, this thesis argues that strategic voting alone does not fully explain deviations from Duverger’s Law observed in India. Instead, it engages with the concept of Zomia, initially introduced by Willem van Schendel and further developed by James Scott, which posits structural differences between highland and plain communities across Asia. Highland societies have historically demonstrated resistance to state integration, fostering more egalitarian and decentralized social systems. The quantitative findings of this research align with Scott's hypothesis, suggesting geography significantly contributes to the observed differences.

Additionally, detailed case studies of Himachal Pradesh and Manipur illustrate how these broader dynamics manifest at the state level. By integrating electoral theory, sociological data, and detailed regional analyses, this thesis provides a novel contribution to the understanding of democratic processes and social structures within India.
