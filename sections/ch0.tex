Since the dawn of human civilization, georaphy has had a profound impact on all facets of human life. It has shaped cultures, traditions, health, economics etc. The abundance of natural oil in the middle east has greatly boosted their economy and led to great development despite the difficult terrain there. Natural resources have also caused great conflicts like the Chincha Islands War fought between spain and its colonies in the 19th century over Chincha Island guano which was dense with nutrients and used as a fertilizer \citep{sheldon2017french}. The war indirectly led to the rise of freedom movement of Spain's colonies against the colonial powers. Geography has also shaped cultures, traditions and spiritual beliefs. For example, regions where rainfall is critical for agriculture many societies developed rituals and beliefs around rain gods  to ensure favourable rainfall for their crops.  In a lot of cultures rivers are considered holy as they provided livelihood for people. They provided fresh water for drinking, agriculture, transportation etc. Examples include Ganges in India, Yangtze in China, Amazon in South America, Nile in Egypt. They've served like cradles of human civilization and human activity has been traced for more than a million years near these rivers. Physical terrain also plays an important role as it helps in development of transportation like railways, roads etc. Construction and transportation over plains has been easier leading to development of significant trade routes like the \textbf{Silk Road} across Asia. The route not only served as a conduit for economic exchanges but also became important as it allowed for exchanges of different cultures from East asia to the Mediterranean world. For example, the presence of Buddhist monasteries led to it being spread across the world. Maritime trade routes also emerged to connect Europe to East Asia which was shortened by the development of the Suez Canal which is now one of the most important strategic location in the world. The development of Silk Road was often hindered by the mountains and were needed to be strategically bypassed. Mountains were often treated with great respect due to there perceived proximity to heaven and were considered divine. Mountains have also served as important nodes for strategic advantage and have remained contested regions for centuries. Mountains, rivers and swamps were natural borders as they were impossible to penetrate hence providing a degree of security and defensibility for political entities. The border between Alberta and British Columbia in Canada roughly follows the Rocky Mountains.  The Himalayas in South East Asia led to a complete separation of the Chinese and Indian diaspora. However, difficult terrain can often lead to isolation of communities and independent development from one other. Examples include the North Sentinel island in India where people have remained isolated and maintained a hunterer gatherer type of lifestyle and are among the world's last uncontacted people. The Sherpa people in Nepal have also lived separately in the Himalayas and over centuries have developed remarkable adaptations including increased efficiency in oxygen utilization making them different from people in plains.  Due to being isolated for centuries, this difference is also reflected in the politics as they developed different political structures. The study of how politics affects geography is called as \textbf{Political Geography}

\section{Politics affecting Geography}
Political geography has profoundly shaped state formation and electoral politics. Throughout history, centralized states have had to contest with communities living in difficult terrains as they formed isolated communities that resisted easy integration. Mountains, deserts, jungles, and far flung islands often became refuges of autonomy  as they lied on fringes of state control. \begin{quote} Mountain people have consistently demonstrated they do not want to live under the rule of outsiders, or often, even share a government with lowlanders\end{quote} 

\hspace*{\fill} - \cite{Hammes2017}. 

As a result many such regions remained only loosely incorporated into pre-modern states. Over time modern states seeking territorial consolidation and national integration have had to devise special policies to incorporate these peripheral areas. Mountain regions historically have been hotbeds of political autonomy. Steep terrain and isolated valleys has allowed highland communities to  resist control by plains. In the Philippines, the Igorot peoples of the Cordillera Mountains successfully resisted Spanish colonization for over three centuries in the northern Luzon \citep{scott1970igorot}. A long struggle ended in the Spanish ultimately failing to conquer these highlands by the end of colonial rule in 1898. Due to difficulty in conquering these regions lowlanders have been forced to enter into negotiations with the mountain people. For example, imperial china  recognized local chieftains (tusi) in the southwestern mountains and allowed them authority in exchange for their allegiance \citep{took2005native}.  

These geographical challenges are immense in India too. Mountain terrains generally have low population density leading to issues of representation. Delimitation exercises need to be carried out carefully as the population densities vary greatly. In Jammu and Kashmir, population density varies from  3,400 people per sq. km in the valley to under 30 per sq. km in the high mountains. 

This has been observed in India too and many scholars have presented qualitative arguments in the difference of behavior of mountains \citep{ali2019delusional,murton2013himalayan,alam2008becoming,hussain2015remoteness}. such societies are called Zomia \citep{van2005geographies}. The idea was introduced by Van Schendel and expanded by Scott in his seminal book ``The Art of not being Governed''. We study how geography has effected all aspects of life, not only in India but throughout the world where different communities have smartly used geography to escape state control. Scott points that the plains and mountains are \textbf{structurally} different from each other. This means that plains and mountains had different religious practices, economic activities and culture. The difference is also seen in it's political structures and gender freedom. Scott presents that women are given more freedom in the mountains than in plains. 