\section{Overview}
Since the dawn of human civilization, georaphy has had a profound impact on all facets of human life. It has shaped cultures, traditions, health, economics etc. In a lot of cultures rivers are considered holy as they provided livelihood for people. They provide fresh water for drinking, agriculture, transportation etc. Examples include Ganges in India, Yangtze in China, Amazon in South America, Nile in Egypt. They've served as cradles of human civilization and human activity has been traced for more than a million years near these rivers. The abundance of natural oil in the middle east has greatly boosted their economy and led to great development despite the difficult terrain there. Natural resources have also caused great conflicts like the Chincha Islands War fought between spain and its colonies in the 19th century over Chincha Island guano which was dense with nutrients and used as a fertilizer \citep{sheldon2017french}. The war indirectly led to the rise of freedom movement of Spain's colonies against the colonial powers. Geography has also shaped cultures, traditions and spiritual beliefs. For example, regions where rainfall is critical for agriculture many societies developed rituals and beliefs around rain gods  to ensure favourable rainfall for their crops.   Physical terrain also plays an important role as it helps in development of transportation like railways, roads etc. Construction and transportation over plains has been easier leading to development of significant trade routes like the \textbf{Silk Road} across Asia. The route not only served as a conduit for economic exchanges but also became important as it allowed for exchanges of different cultures from East asia to the Mediterranean world. For example, the presence of Buddhist monasteries led to Buddhism being spread across the world. Maritime trade routes also emerged to connect Europe to East Asia which was shortened by the development of the Suez Canal which is now one of the most important strategic location in the world. The development of Silk Road was often hindered by the mountains and were needed to be strategically bypassed. Mountains were often treated with great respect due to there perceived proximity to heaven and were considered divine. Mountains have also served as important nodes for strategic advantage and have remained contested regions for centuries. Mountains, rivers and swamps were natural borders as they were impossible to penetrate hence providing a degree of security and defensibility for political entities. The border between Alberta and British Columbia in Canada roughly follows the Rocky Mountains.  The Himalayas in South East Asia led to a complete separation of the Chinese and Indian diaspora. However, difficult terrain can often lead to isolation of communities and independent development from one other. Examples include the North Sentinel island in India where people have remained isolated and maintained a hunterer gatherer type of lifestyle and are among the world's last uncontacted people. The Sherpa people in Nepal have also lived separately in the Himalayas and over centuries have developed remarkable adaptations including increased efficiency in oxygen utilization making them different from people in plains.  Due to being isolated for centuries, this difference is also reflected in the politics as they developed different party structures. The study of how politics affects geography is called as \textbf{Political Geography}

\section{Politics affecting Geography}

Political geography has profoundly shaped state formation and electoral politics. These geographical challenges are immense in India too as accessibility becomes a big issue. Mountain terrains generally have low population density leading to issues of representation. Delimitation exercises need to be carried out carefully as the population densities vary greatly. In Jammu and Kashmir, population density varies from  3,400 people per sq. km in the valley to under 30 per sq. km in the high mountains \citep{kumar2022measuring}. Mountain communities often depend on niche economies like tourism, horticulture or government jobs. Hills often have lower voter turnout too. During the 2017 Uttarakhand assembly elections, hill districts like Tehri (55.68\%), Pauri (54.86\%), and Almora (53.07\%) recorded significantly lower turnouts compared to the state's average of 65.6\%. This creates issues as often some people ``left behind'' in terms of development. The economies in hills are weak and often need basic amenities like road, water, jobs and electricity. These also become the political issues in the mountains. Recent elections have shown how increase in road network have increased chances of getting re elected in \textit{rural india} \citep{basistha2024elections}. However, it is important to note that traditionaly mountains also served as zones of resistance. 

 Throughout history, centralized states have had to contest with communities living in difficult terrains as they formed isolated communities that resisted easy integration. Mountains, deserts, jungles, and far flung islands often became refuges of autonomy  as they lied on fringes of state control. \begin{quote} Mountain people have consistently demonstrated they do not want to live under the rule of outsiders, or often, even share a government with lowlanders\end{quote} 

\hspace*{\fill} - \cite{Hammes2017}. 



Many such regions remained only loosely incorporated into pre-modern states. Over time modern states seeking territorial consolidation and national integration have had to devise special policies to incorporate these peripheral areas. Mountain regions historically have been hotbeds of political autonomy. Steep terrain and isolated valleys has allowed highland communities to  resist control by plains. In the Philippines, the Igorot peoples of the Cordillera Mountains successfully resisted Spanish colonization for over three centuries in the northern Luzon \citep{scott1970igorot}. A long struggle ended in the Spanish ultimately failing to conquer these highlands by the end of colonial rule in 1898. Due to difficulty in conquering these regions lowlanders have been forced to enter into negotiations with the mountain people. For example, imperial china  recognized local chieftains (tusi) in the southwestern mountains and allowed them authority in exchange for their allegiance \citep{took2005native}.  The case of the Nuba Mountains in Sudan provides a compelling example of how geographical isolation can create a strong collective identity among diverse tribal groups. It indicates that mountainous regions are susceptible to formation of regional political parties which cater to their unique interests and identity due to their geography.

This has been observed in India too and many scholars have presented qualitative arguments in the difference of behavior of mountains \citep{ali2019delusional,murton2013himalayan,alam2008becoming,hussain2015remoteness}. Such societies are called Zomia \citep{van2005geographies}. The idea was introduced by Van Schendel and expanded by Scott in his seminal book ``The Art of not being Governed''. We study how geography has effected all aspects of life, not only in India but throughout the world where different communities have smartly used geography to escape state control. Scott argues that plains and mountains had different religious practices, economic activities and culture. The difference is also seen in it's party structures and gender freedom. Scott presents that women are given more freedom in the mountains than in plains. In the end, Scott points that the plains and mountains are \textbf{structurally} different from each other. This leads to our research questions.

\section{Research Questions}

\begin{enumerate}
    \item \textbf{How do the political systems of mountainous and plain regions in India differ, and how have these differences evolved over time? }
    \item \textbf{How do gender differences in mountainous and plain regions of India differ, and how have these differences evolved over time?}
    \item \textbf{What qualitative theories account for the political and gender expressions between mountainous and plain regions?}
\end{enumerate}

To study this we employ a mixed methods approach and use both quantitative and qualitative approaches to study the questions. This will help us cover significant depth and breadth in the problem. To study the politics of both mountains and plains we use party structure in the country as a proxy to analyse. Dominant political parties often serve as a reflection of the ideology of common people \citep{romeijn2020political}. By studying the dominant political parties of each district, we can see how different the regions are different politically. Party structure of national parties a country is a broad theme which can be operationlised in different ways. It can be studied by looking at ideologies of parties, member of parties, electoral performance, existence of formal party symbols etc. All of these will tell us about different facets of a country. By studying ideology of parties, we can find the spectrum of political thought within the country. Prevalence of centrist parties indicates a political culture that favors moderation and vice versa. Studying the membership composition can tell us  which segments of society align with particular parties depending on their age, ethinicity, socioeconomic status etc. Logos reflect how parties communicate there ideas to people. Logos can be deep embedded in culture, history etc and show themes among general public. Analysing electoral performance has been the most common way of judging a party. It tells us which regions align with a party and national support suggests a party's broader appeal. Analyzing electoral performance over time can also indicate shifts in public opinion. In this study, we analyse the electoral performance albeit in a different way. We operationalise the electoral results using Duverger's law which has been a central law in politics for decades and predicted rise and fall of party systems for decades. 

\vspace{0.2cm}

Our second research question focuses on the differences in freedom of gender expression for both the regions. Scott argues that mountains and remote societies allow for more freedom and expression for women than the plain societies which are under strict hierarichal structures. We use a combination of unique parameters from NFHS dataset (National Family Health Survey) which are not used together before and combine them together to study how much women get support from families, financial independence, education, bodily autonomy etc. These multitude of factors will help us verify our hypothesis.

\vspace{0.2cm}

In the end we investigate the possible reasons for these differences. It is impossible to establish causation for the given results without further detailed quantitative analysis for which the data is currently missing. However, we discuss the \textit{\textbf{possible}} reasons for our results and dig deep in literature for scholars who have found similar ideas not only in India but across the world. As discussed above, Zomia is one of the possible reasons for the same. The explanations can vary from definitional variations of Duverger's law to the idea of Zomia and beyond. 

\section{Challenges Faced}

\textbf{Data Collection}: This study involved scraping data from Election Commission of India \citep{ECI_WEBSITE} website which is unscrapeable after a few attempts. To bypass this, we used a web browser simulator known as \textbf{Selenium}. Selenium is a python library widely used for web scraping, automated testing, and repetitive browser tasks. It provides functionality for web scraping, automated testing, and repetitive browser tasks. The ECI provides data for older elections in PDF format which required use of python libraries to scrape and collect data. After scraping, a few constituencies had different names for different years. For example, NAINITAL was named as NAINATAL (missing an I). For this we use fuzzy word matching which uses levenshtein distance to calculate the distance between words. The data was compiled after manual verification for each state.

\section{Thesis Overview}

\begin{itemize}
    \item \textbf{Chapter 1} (current chapter): Provides a base for the study and introduces readers to the background required for the study.
    \item \textbf{Chapter 2}: The second chapter contains the literature review of the thesis, which discusses the history of electoral politics in India. It also conducts a specific review of electoral politics in the Northern Mountains i.e. Himalayas of India. It also provides a detailes review of the development of Duverger's law, not only in India but across the world. This helps to lay foundation for the electoral performance of parties in India. We also look how mountains and plains have been structurally different across the entire world. We look at this in India by studying \textbf{zomia} in detail and study what various other authors presented about it.
    \item \textbf{Chapter 3:} The third chapter aims to answer the first two research questions and is divided in two halves. The first half tries to explain how Duverger's Law, works in India's mountain and plain states. We also look at whether the mountain regions in India have some structural political differences from the Indo Gangetic plains by analyzing the electoral trends from 1977 to 2014. The second half focusses on the differences in gender expression. The chapter presents the methologies in detail and verifies the hypothesis of Zomia by using quantiatitative approaches.
    \item \textbf{Chapter 4:} The fourth chapter focuses on the third research question. We emphasize on the plausible reasons like strategic voting, identity politics and Zomia. We identify that the areas of identity formation often result from the resistance against the centralized power, as in the case of the formation of the Pahari identity in Uttarakhand as well as ethnic conflicts in Manipur. Also structural difference between plains and mountainous society in terms of the economic role of women, patriarchy and kinship structures are highlighted. The chapter also compares how these characteristics are integrated or marginalized by post-colonial nation states like India and Pakistan. To conclude, we take case studies of Himachal Pradesh and Manipur to analyse how these changes are not just limited on a national level and can be seen at minor state differences.
    \item \textbf{Chapter 5:} This is the concluding chapter which summarises the key insights like how the study  has attempted to understand the applicability of Duverger's law within the Indian context in light of state based social cleavages, geographical isolation and political autonomy shaping electoral outcomes in varying degrees in different states. In the end we discuss the future scope for our work.
\end{itemize}