\section{Introduction}
Political geography has profoundly shaped state formation and electoral politics. Throughout history, centralized states have had to contest with communities living in difficult terrains as they formed isolated communities that resisted easy integration. Mountains, deserts, jungles, and far-flung islands often became refuges of autonomy or rebellion as they lied on fringes of state control. \begin{quote} Mountain people have consistently demonstrated they do not want to live under the rule of outsiders, or often, even share a government with lowlanders\end{quote} 

\hspace*{\fill} - \cite{Hammes2017}. 

As a result many such regions remained only loosely incorporated into pre-modern states. Over time modern states seeking territorial consolidation and national integration have had to devise special policies to incorporate these peripheral areas. They've tried to introduce 