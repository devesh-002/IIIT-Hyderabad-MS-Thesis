\begin{sloppypar}

\section{Overview}
Geography has played a pivotal role in shaping human societies across the world. It has shaped cultures, traditions, health, economics etc. Rivers like the Ganges, Yangtze, Amazon provided fertile soil for the earliest agricultural settlements that grew into complex civilizations. However, rugged mountains and harsh deserts limited settlements and provided a natural fortress to the people who lived there. These terrains often determined the connection to other communities like fertile and arable lands provided by river valleys fostered large scale trade and network. High mountains, deserts on the other hand acted as barriers and isolated settlements for centuries. For example, the Taurus Mountains long kept Anatolia apart from the rest of Asia, much as the Atlas Mountains wall off North Africa. Mountains, rivers and swamps were natural borders as they were impossible to penetrate hence providing a degree of security and defensibility for political entities. Even in the modern day, these terrains behave as borders separating huge nation states. For example, the border between Alberta and British Columbia in Canada roughly follows the Rocky Mountains. 

\vspace{0.3cm}

Geography also shaped the economic life of civilizations as abundant natural resources meant prosperity and extensive trade. Historically, regions with calm coastlines, riverbanks became important nodes of trade across the world. Economists like \cite{smith1937wealth} have argued how extensive trade, prosperity and bustling economy led to centralization of power and hierarichal structures. They presented that large economic structures cannot survive without a central power \citep{robinson2012nations}. In lowland plains, societies often developed intensive labor agriculture leading to a great demand for labor. Organization of labor became necessary, and jobs needed to be divided. The elites could extract resources (grain, labor, taxes) from a concentrated population to build armies and bureaucracies. For example, The Kingdom of Kongo was formed to control the natural resources and imposed heavy taxes on the working populous and engaged in slave trade too. The mountain societies on the other hand, due to remote geographies didn't require these hierarichal structures. The population was more scattered, spread and,  steep slopes, deep valleys, harsh winters, limited arable land imposed a cap on the agricultural surplus on these societies. Incorporating them in central structures was less profitable and exponentially more difficult due to the challenging terrain which needed to be traversed. Hence, the political structures in highlands were based on local autonomy and were more kinship based.

\vspace{0.3cm}

However, historically difficult terrains not only hindered state control but also influenced identity and representation. They were isolated from the plains in their own valleys which led to formation of their own unique identities, languages etc. These became refuges for cultures or religions different from those dominant in the plains. For example, minority groups such as the Maronites in Lebanon, the Kurds in the Zagros Taurus ranges, or the Alawites in Syria historically retreated to the mountains and sustained unique identities.
 Throughout history, centralized states have had to contest with communities living in difficult terrains as they formed isolated communities that resisted easy integration. Benedict Anderson in his seminal work ``Imagined communities'' elaborates on the conflicts between the plains and mountains. He argues that a central identity is important for building a modern nation state and these hill communities were often resistant to the idea due to their unique identities. He explains how the Thai government didn't allow development of writing systems and literature for hill tribe minority languages as this would preserve their identity which was seen as a threat to national unity \citep{anderson1991imagined}. This would make them difficult to incorporate with the mainlands.  \begin{quote} Mountain people have consistently demonstrated they do not want to live under the rule of outsiders, or often, even share a government with lowlanders\end{quote} 

\hspace*{\fill} - \cite{Hammes2017}. 



Many such regions remained only loosely incorporated into pre-modern states. Over time modern states seeking territorial consolidation and national integration devised special policies to incorporate these peripheral areas. Steep terrain and isolated valleys has allowed highland communities to  resist control by plains. In the Philippines, the Igorot peoples of the Cordillera Mountains successfully resisted Spanish colonization for over three centuries in the northern Luzon \citep{scott1970igorot}. A long struggle ended in the Spanish ultimately failing to conquer these highlands by the end of colonial rule in 1898. Due to difficulty in conquering these regions lowlanders have been forced to enter into negotiations with the mountain people. For example, imperial china  recognized local chieftains (tusi) in the southwestern mountains and allowed them authority in exchange for their allegiance \citep{took2005native}. Nuba Mountains in Sudan provides explains how geographical isolation can create a strong collective identity among diverse tribal groups. It indicates that mountainous regions are susceptible to formation of regional political parties which cater to their different interests from the plains and identity due to their geography.

This has been observed in South Asia too and many scholars have presented qualitative arguments in the difference of behavior of mountains \citep{ali2019delusional,murton2013himalayan,alam2008becoming,hussain2015remoteness}. Such societies were called Zomia \citep{van2005geographies}. The idea was introduced by Van Schendel and expanded by Scott in his seminal book ``The Art of not being Governed''. They studied how geography has effected all aspects of life, not only in India but throughout the world where different communities have smartly used geography to escape state control. Scott argues that plains and mountains had different religious practices, economic activities, culture, party structures and gender expression. Hence, Scott points that the plains and mountains are \textbf{structurally} different from each other.  In the modern era, we have observed in India how hills often have lower voter turnout compared to plains too. During the 2017 Uttarakhand assembly elections, hill districts like Tehri (55.68\%), Pauri (54.86\%), and Almora (53.07\%) recorded significantly lower turnouts compared to the state's average of 65.6\%. Remote terrain also leads to less development and modern amenities. This creates issues as often some people ``left behind'' in terms of development. The economies in hills are weak and often need basic amenities like road, water, jobs and electricity. These also become the political issues in the mountains. Recent elections have shown how increase in road network have increased chances of getting re elected \citep{basistha2024elections}. These issues are also present in plains but, they also focus on identity politics, as we will see later. Hence, the agendas of politics are also different in plains and mountains.

\vspace{0.3cm}

The above ideas are seen under the umbrella term of ``Political Geography''. \cite{kitchin2009international} define political geography as ``Geographical study of electoral systems and results but can, in a broader sense, relate to the varied processes through which spatial change is sought in, with, by, or to places.'' This thesis will focus on the Political Geography in India and see how geographical difference has led to different identities, cultures, economies which in turn \textit{might} affect the political structures in plains and mountains. This leads to our research questions.

\section{Research Questions}

\begin{enumerate}
    \item \textbf{How do the political systems of mountainous and plain regions in India differ quantiatitatively, and how have these differences evolved over time? }
    \item \textbf{How do gender differences in mountainous and plain regions of India differ quantiatitatively, and how have these differences evolved over time?}
    \item \textbf{What qualitative theories account for the political and gender expressions between mountainous and plain regions?}
\end{enumerate}

To study this we employ a mixed methods approach and use both quantitative and qualitative approaches to study the questions. This will help us cover significant depth and breadth in the problem. While a lot of Anthropologists and Political scientists have studied it qualitatively, we attempt to do so quantitatively. To study the politics of both mountains and plains we use party structure in the country as a proxy to analyze. Dominant political parties often serve as a reflection of the ideology of common people \citep{romeijn2020political}. By studying the dominant political parties of each district, we can see how different the regions are different politically. Party structure of national parties a country is a broad theme which can be operationalized in different ways. It can be studied by looking at ideologies of parties, member of parties, electoral performance, existence of formal party symbols etc. All of these will tell us about different facets of a country. By studying ideology of parties, we can find the spectrum of political thought within the country. Prevalence of centrist parties indicates a political culture that favors moderation and vice versa. Studying the membership composition can tell us  which segments of society align with particular parties depending on their age, ethnicity, socioeconomic status etc. Logos reflect how parties communicate there ideas to people. Logos can be deep embedded in culture, history etc and show themes among public. Analyzing electoral performance has been the most common way of judging a party. It tells us which regions align with a party and national support suggests a party's broader appeal. Analyzing electoral performance over time can also indicate shifts in public opinion. In this study, we analyze the electoral performance albeit differently. We operationalize the electoral results using Duverger's law which has been a central law in politics for decades and predicted rise and fall of party systems for decades. 

\vspace{0.2cm}

Our second research question focuses on the differences in freedom of gender expression for both the regions. Scott argues that mountains and remote societies allow for more freedom and expression for women than the plain societies which are under strict hierarichal structures. We use a combination of unique parameters from NFHS dataset (National Family Health Survey) which are not used together before and combine them together to study how much women get support from families, financial independence, education, bodily autonomy etc. This multitude of factors will help us verify our hypothesis.

\vspace{0.2cm}

In the end we investigate the possible reasons for these differences. It is \textbf{not possible} to establish causation for the given results without further detailed quantitative analysis for which the data is currently missing. However, we discuss the \textit{\textbf{possible}} reasons for our results and dig deep in literature for scholars who have found similar ideas not only in India but across the world. As discussed above, Zomia is one of the possible reasons for the same. The explanations can vary from definitional variations of Duverger's law to the idea of Zomia and beyond. 

\section{Challenges Faced}

\textbf{Data Collection}: This study involved scraping data from Election Commission of India \citep{ECI_WEBSITE} website which is un-scrapable after a few attempts. To bypass this, we used a web browser simulator known as \textbf{Selenium}. Selenium is a python library widely used for web scraping, automated testing, and repetitive browser tasks. It provides functionality for web scraping, automated testing, and repetitive browser tasks. The ECI provides data for older elections in PDF format which required use of python libraries to scrape and collect data. After scraping, a few constituencies had different names for different years. For example, NAINITAL was named as NAINATAL (missing an I). For this we use fuzzy word matching which uses Levenshtein distance to calculate the distance between words. The data was compiled after manual verification for each state.

\section{Thesis Overview}

\begin{itemize}
    \item \textbf{Chapter 1} (current chapter): Provides a base for the study and introduces readers to the background required for the study.
    \item \textbf{Chapter 2}: The second chapter contains the literature review of the thesis, which discusses the history of electoral politics in India. It conducts a review of electoral politics in the Northern Mountains i.e. Himalayas of India and also provides a detailed review of the development of Duverger's law, not only in India but across the world.  We also look how mountain and plain societies are structurally different across the entire world. We focus on India by studying \textbf{Zomia} in detail and study what various other authors presented about it.
    \item \textbf{Chapter 3:} The third chapter aims to answer the first two research questions and is divided in two halves. The first half tries to explain how Duverger's Law works in India's mountain and plain states. We also look at whether the mountain regions in India have some structural political differences from the Indo-Gangetic plains by analyzing the electoral trends from 1977 to 2014. The second half focuses on the differences in gender expression. The chapter presents the methodologies in detail and verifies the hypothesis of Zomia by using quantitative approaches.
    \item \textbf{Chapter 4:} The fourth chapter focuses on the third research question. We emphasize on the plausible reasons for the difference between mountain and plain societies. These include  strategic voting, identity politics and Zomia. We identify that the areas of identity formation often result from the resistance against the centralized power, as in the case of the formation of the Pahari identity in Uttarakhand as well as ethnic conflicts in Manipur. Also structural difference between plains and mountainous society in terms of the economic role of women, patriarchy and kinship structures are highlighted. The chapter also compares how these characteristics are integrated or marginalized by post-colonial nation states like India and Pakistan. To conclude, we take case studies of Himachal Pradesh and Manipur to analyze how these changes are not just limited on a national level and can be seen at minor state differences.
    \item \textbf{Chapter 5:} This is the concluding chapter which summarizes the key insights like how the study  has attempted to understand the applicability of Duverger's law within the Indian context in light of state based social cleavages, geographical isolation and political autonomy shaping electoral outcomes in varying degrees in different states. In the end we discuss the future scope for our work.
\end{itemize}

\end{sloppypar}