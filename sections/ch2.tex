\section{Introduction}

 // write more in this on how the chapter expands

\section{Historical Case}
Mountain regions historically have been hotbeds of political autonomy. Steep terrain and isolated valleys has allowed highland communities to  resist control by plains. In Southeast Asia’s highlands (Zomia) upland tribes have remained beyond the full reach of valley kingdoms. In the Philippines, the Igorot peoples of the Cordillera Mountains successfully resisted Spanish colonization for over three centuries in the northern Luzon \citep{scott1970igorot}. A long struggle ended in the Spanish ultimately failing to conquer these highlands by the end of colonial rule in 1898. Due to difficulty in conquering these regions lowlanders have been forced to enter into negotiations with the mountain people. For example, imperial china  recognized local chieftains (tusi) in the southwestern mountains and allowed them authority in exchange for their allegiance \citep{took2005native}. // expand more?

\section{Can Strategic Voting explain this?}

Strategic voting is often used to explain Duverger’s law as voters do not vote for their most preferred candidate or party but rather for a less-preferred option if that has a better chance of winning. India’s case is different as there is a mixed evidence of strategic voting. 

\vspace{0.3cm}

Chhibber uses SF ratio to show that strategic voting happens in masses in India. SF ratio is the ratio of votes obtained by the second loser to the first loser. Chhibber argues that if SF ratio is near 0 then there is strategic voting happening at significant levels and vice versa. However, SF ratio is argued not to be a reliable metric as Diwakar mentions 

\vspace{0.3cm}

“SF ratio close to 1 (signifying a non-Duvergerian equilibrium) is possible in two situations: first, where the winning party secures a large majority of votes, leaving a very small proportion of votes for the other parties, and, second, where many parties share the votes in a closely fought election.”

\vspace{0.3cm}

Hence, SF ratio is not a reliable metric. While some studies show that Indian voters tend to be more strategic than expressive when their preferred party is unlikely to win \citep{choi2009strategic}, a recent study showed modest evidence for elite collusion explaining the voting patterns in India and also showed that strategic voting is practically absent in India \citep{ziegfeld2021accounts}. A case study on Uttar Pradesh \citep{heath2022so} using survey data instead of statistical methods showed that there was no evidence for strategic voting and the majority of the people expected their party to win. The study was limited in scope due to its small sample set and evidence from one state only and Uttar Pradesh doesn't necessarily follow Duverger's law but it showed how the metrics used to evaluate strategic voting were weak and it was first of its kind to use survey data. 

\vspace{0.3cm}

Hence, it can be argued that strategic voting cannot be the case necessarily in these states. India unlike other countries is very diverse with a very complex social fabric with various identities like caste, gender, tribal groups, religion, geography etc coming into play and has different reasons for divergence from Duverger's law apart from strategic voting. The case of a single identity being the reason behind explaining Duverger’s law is not a new idea \citep{mayer2013gross} for India. Mayer analysed parameters like dummy candidates, spoiler candidates, regional parties etc and found the ratio of SC/ST i.e. caste and identity being a moderate reason behind the metric of effective parties. 

Moving forward we try to present  to explain how geography has effected politics in history and India. It is important to note that this can be one of the \textbf{possible} reasons, we are not directly drawing a causation based on the correlation and are just mentioning the theories that have argued for this in and outside India.


\section{Indian Case}
The division between mountain and plain societies has been evident in the Indian subcontinent since the colonial era. The British Empire in India governed the mountainous Northwest Frontier (today’s Pakistan-Afghanistan border) through indirect means. British enforced the Frontier Crimes Regulations via local Pashtun maliks rather than imposing regular law. This meant more local autonomy for the Pashtun chieftains and acknowledged the difficulty of Britishers in gaining control \citep{ali2013indigenous}. This colonial arrangement remained in independent Pakistan as the Federally Administered Tribal Areas (FATA). This was a longstanding issue and  Pakistan finally merged FATA into Khyber Pakhtunkhwa in 2018 after years of struggle and militancy \citep{horgan2008leaving}. A similar issue can be observed in the Jammu \& Kashmir state of India where a special status under the Article 370 was granted to the state which was abruptly abolished in 2019. States have been forced to give away there autonomy and give local exemptions to incorporate these regions in the state. Article 371-A in India, gives Nagaland state control over its customary laws and land rights. Sikkim has been given special exemption from income tax laws as Sikkim operated under its own tax laws before it was merged into India in 1975. These exemptions were preserved under Article 371F of the Indian Constitution. Similarly, autonomous constituencies are present in  Assam, Manipur and Ladakh to empower indigenous communities and ensure self-governance \citep{sixthschedule}. These councils can make laws on subjects such as land use, forest management, agriculture and village administration. They also have the authority to establish courts for cases involving tribal members, provided the sentences do not exceed five years of imprisonment. Apart from the provisions in the constitution there have been electoral difference in the voting patterns as explained in the results in the previous chapter. 

\section{Role of Identity politics in Mountains}

The formation of identities is a continuous process often done as a response to external stimuli. The rise of different identities in the mountains and plains also points towards the difference in there behaviors too. Uttarakhand and Himachal saw a rise of distinct ``pahari" identity \citep{mishra2000politics} which was different from the rise of caste politics in the plains. This is despite the fact that caste based cleavages existed in mountains too. Almost 50\% of Himachal Pradesh consists of Rajput and Brahman's combined. After Uttarakhand's formation, it was noted that statehood actually accentuated the Pahar Maidan divide within Uttarakhand itself \citep{mathur2015remote}. In recent news too Dehradun, Haridwar, and Udham Singh Nagar (plain districts of Uttarakhand) have experienced significant infrastructural and economic growth rather than the remote areas \citep{Mohammad_TOI}. The formation of different identities in North East might seem similar to the caste cleavages present in the northern plains. However, it is important to note that these identities emerged as a result of geographic differences. The conflict between the hill and plains communities served as a crucial catalyst in the formation of distinct identities, particularly for the hill tribes who sought to differentiate themselves from the perceived cultural and political dominance of the plains populations. Due to the demand of Assamese as an official language the hill tribes especially felt threatened \citep{inoue2005integration}. Hills were used as tool to resist against central powers as Nagas hid in the Naga Hills-Tuensang Area and Patkai mountain \citep{kapai2020spatial}. Due to these mountains being porous, it allowed for a 1600+ border to be open with Myanmar as Indian officials have officially acknowledged the difficulty in closing the same \citep{Bureau_2024}. Scholars have argued the efficacy of statehood and autonomy arrangements.  B. M. Pugh was a Khasi leader who was in favour of creating a state consisting of hill areas only. He wished to combine present day Meghalaya, Nagaland, Arunachal Pradesh, Mizoram together but the motion didn't go through as the central government feared divisions on the basis of religous lines i.e. formation of a Christian state \citep{karlsson2013evading}. The above discussion shows how the identities were developed as a result of conflict from hills. However, this led to the rise of ethnic conflicts among these tribes. The most violent can be Kuki and Naga as tensions rose in Manipur.  Examples also include the long-standing demands for a separate state of Bodoland within Assam  and the autonomy movement of the Karbi community \citep{sarma2017bodoland}. Economic disparities also played an important role as hill tribes were often neglected and received minimal support making them feel alienated from the mainland India. Economic neglect was an important factor contributing to the rise of the United Liberation Front of Assam (ULFA) insurgency \citep{chima2023insurgency}. It is interesting to note what caused the rise of a seperate strong hill identity across multiple states despite presence of strong ethnic divisions which effected the rest of India. This phenomenon is not uncommon to India and is seen throughout the world in ``Maroon communities"  or ``undercommons" referenced in the literature review. India's case of Zomia and its consequences will be elaborated more in the next section. 


\section{Zomia}
 Zomia is a term coined in 2002 by Dutch scholar Willem van Schendel to describe a vast highland region on the fringes of South and Southeast Asia. The name derives from Zomi, meaning “highlander” in local Tibeto-Burman languages \citep{van2005geographies}. 
 The evidence of mountain societies being structurally different from plains was seen in India where Van Schendel \citep{van2005geographies} presents the idea of Zomia in which he argues that the borders drawn between major states are arbitrary and were without consideration of the social/cultural boundaries. 
Scott \citep{jamesscott} elaborates on this and extends the existence of the Zomia framework, a stateless society which was the last escape zone and resisted incorporation into the power of state. In modern day, the Zomia region consists of the Mountains in North, North east of India, Tibet and mountainous regions of South east Asia (Himalayan mastiff). Initially the central Himalayas were not a part of this framework but studies  show that the Zomia framework can be used to explain the Central Himalayas \citep{shneiderman2010central}. These remote areas in the Himalayan mastiff were often used to exile unwanted people due to religious and ethnic conflicts but Scott argues that it was actually the opposite. The majority people in these societies deliberately left the state in order to escape it and do trade without any restrictions and escape the crutches of hierarchical divisions and feudal governments to form more egalitarian governments which gave more freedom to women too. These areas were important passes and present on international routes and hence could be easily controlled. Due to their importance, attempts were made in history by kingdoms to incorporate them into states but mostly backfired due to the extreme remote nature of these districts. Often these areas were ignored by scholars due to the remoteness and lack of documented history especially in the Chinese side of the Himalayas which has very strict rules for journalists and data collection. All these restrictions are increased due to the lack of knowledge about the language making it a difficult but important region to study. Although labeled backward/tribal by the state due to their limited history, Scott argues that they have deliberately avoided writing and not have written records. The oral history of these areas becomes an important aspect of study for us. He argues that such states tend to be politically different from the mainland and are egalitarian and free of the crutches of hierarchy like caste which is prevalent in mainlands of India. Although remote, this region has been a very important strategic location due to India and China in close contest against each other in this region who are trying to win over the locals to gain control over important geographical points  and more natural resources like Brahmaputra river and its massive basin. The idea to control eastern himalayas was conceived by the British but due to remoteness, the eastern Himalayas were the last regions not to be captured and trade routes were established to China through current day Assam along the Inner Line. The Inner Line was established in the Eastern Himalayan frontier to regulate movement and interaction between the plains of Assam and the tribal areas of the hills. Post independence, the NEFA (North-East Frontier Agency), present day Arunachal Pradesh was a contested territory between India and China. Both the states were trying to appease the locals and establish control \citep{guyot2017shadow}. This control is often achieved by building \enquote{spheres of influence} around important nodes and grow them \citep{Farrelly_2013b}. In India, Miao in Arunachal Pradesh is an important node of control for the state to gain access to the otherwise remote region. Even though remote there are tools like all season roads, circuit houses and especially schools and colonies of government officials used by government to spread its influence. District collectors are often appointed from the state of Arunachal or Assam to woo the locals which provides the much needed local support and legitimacy to the Indian government.  

\vspace{0.3cm}

 \enquote{The region has never been united politically, neither as an empire nor as a space shared among a few feuding kingdoms, nor even as a zone with harmonized political systems. Forms of distinct customary political organizations, chiefly lineage-based versus \enquote{feudal} unlike plains where feudal systems developed and were controlled by a small elite. They subjugated egalitarian groups in their orbit, but never united, and were never totally integrated into surrounding polities.} - \citep{michaud2017s}

\vspace{0.3cm}

 Even though the existence of Zomia shows political distinctiveness from plains, Scott presents that this was till 1950 only. After that with technological innovations and Zomia becoming a contested area as it became part of borderlands, states quickly developed to incorporate them into their structure and the Zomia came to an end. However with development of inter-border roads scholars have presented that this might have reopened the debate of Zomia as the state built infrastructure facilitates the movement between these areas opening new opportunities for these markets and areas \citep{murton2013himalayan}. 

 \vspace{0.3cm}

 These instances of marginalization and exclusion are not confined to Indian territory. They are also evident in Gilgit-Baltistan, a mountainous region in Pakistan where the local population has often felt disconnected from mainland Pakistan. Among the natives of the region there is a sense of betrayal or \enquote{khelna} against the Pakistani government due to the neglect and systemic exclusion. Scholars and authors have highlighted that this divide is partly rooted in religious-sect differences as Gilgit-Baltistan is predominantly Shia while the rest of Pakistan is  Sunni-dominated. However, it is also argued that geography plays a significant role in this dynamic. The region's remote, rugged terrain has  made it difficult to integrate fully into the national framework  increasing the feeling of isolation and alienation \citep{ali2019delusional}. Similarly, in the Hunza region of Northern Pakistan  remoteness is not just a geographical condition but it has been used as a strategic tool for bargaining. Perceived as savage and dangerous by the colonialists during the British occupation (similar to Zomia), it has used remoteness to promote tourism in the region extensively as a \enquote{lost paradise}. It has also become an important strategic location and due to technological  advances it now has satellite guided missiles, highways etc \citep{hussain2015remoteness}. This is similar to what happened to remote regions in India, especially the Arunachal Pradesh.

\vspace{0.3cm}

Geography affecting politics can be seen in the modern day as well and its not necessary for the terrains to be vastly different, even small variations like hills and its valleys can show a difference. It is also not necessary for it to happen at a macro level, differences among people due to geography can be seen at a state level in India. In this section we will explain a few case studies in India where this difference is present. 

\section{Structural Differences}

Zomia has also presented on how the plains and mountains are different. In this section we will focus on the gender and political differences of the mountains and plains. Scott has argued that mountain socities are more egalitarian and in some cases also have matrilineal kinship patterns. This is in contrast with the plain socities which are starkly different d ue to there patrilineal and male-centered lineage systems. This is clearly evident in the \textbf{kinship and family structures of the mountain societies}. For example, in Meghalaya the Khasi and Garo tribes are matrilineal. The youngest daughter inherits all ancestral property, children bear the mother’s surname and husbands move into the wife’s maternal home after marriage \citep{Allen_2012}. Lowland state societies influenced by Confucian, Hindu or Islamic law tended to formalize male authority in family and property matters (e.g. requiring sons for inheritance, emphasizing female chastity and patrilocal marriage. These kinship laws have been there for centuries and are still seen in the modern day. It is important to note that this was not limited to kinship laws only. Women’s economic roles in mountain communities have generally been as prominent as men’s. Women would take charge of household gardens and tend to them. As highland societies were mostly self reliant, women's labor was more visible and indispensable giving them a better economic status. This is not only visible in the Khasi society but it is also seen in the Lahu people of southwest China (a Tibeto-Burman hill group). Lahu men and women share responsibilities in farming, decision-making,  and a married couple is considered a single social unit in community affairs \citep{Du_2015}. This egalitarian labor partnership has persisted among the Lahu despite pressures from the patriarchal Chinese state over the last two centuries. Plains have often restricted the role of women to childbearing, homes and informal markers which makes there value invisible in the economic markets. Additionaly, premarital sex by women might be socially tolerated and women can remarry without stigma. This is seldom allowed in orthodox lowland cultures that emphasized female chastity and one-time marriage alliances (often for political or economic gain)
 Men generally hold position of responsibilities in public spaces, politics etc. The power dynamics between genders in upland societies tend to be more fluid, with women often having greater informal influence than in lowland patriarchies. Some highland cultures even allow women to serve as clan heads or spiritual leaders.  For example, numerous ethnic groups in the Southeast Asian Massif have traditions of female shamans or oracles who guide communal spiritual life. State-sanctioned religions often excluded women from leadership (e.g. only men could become Buddhist monks of high rank or Confucian scholars). In valley societies it is important to note the difference between matrilineal and matriarchy. In the above mentioned example of matrilineal societies of Khasi hills, men were often allowed to control the the village councils (dorbar) \citep{WashingtonPost_2015,TheGuardian2011}. In lowland societies women’s autonomy was often curbed by stricter marriage customs, purdah or seclusion practices (in some Hindu and Muslim kingdoms). This was often accompanied by the expectation to obey fathers and husbands codified in law \citep{Papanek_1973,Devi2019}.


As Scott argued that plains and mountains are structurally different, social differences like difference between kinship and family structures, economic freedom of women and matrilineal social norms were not the only differences observed between the Zomia and lowland areas. These regions also diverged in their political structures and hierarchies. In the Zomia framework, mountain societies tend to be more decentralized, mobile, and egalitarian, while plain societies are more settled. Plain societies are characterized with more centralized authority and rigid hierarchies \citep{Hammond_2011}.

// add point of zomia people being away from the plains

Political organization in the hills tends to be local and kinship-based (village councils, clan elders, tribal chiefs) rather than the hierarichal structures and a top down bureaucracy. Many highland groups formed ``egalitarian or acephalous" communities with no permanent chiefs or with only weak leadership roles. Scholars have argued that this was a deliberate choice among these societies. In general, Zomia’s highlanders ``paid neither taxes to monarchs nor regular tithes to a permanent religious establishment”. This is opposite to the valley people who were often forced to pay taxes to the church and the state. Local moneylenders often worsened the situation by employing brute force to collect taxes, which were considered a birthright of the monarchy. For example, the jizya tax was imposed on non-Muslims and the development of the zabt system and the dahsala taxation method during the medieval period \citep{moosvi1973production}. Highland societies often favoured mobility which led to the development of practicing shifting cultivation or the slash and burn agriculture. These crops are easy to scatter and harvest at different times, making it hard for would-be tax collectors to confiscate a single big harvest. The lowland states depended on intensive agriculture (irrigated rice, canals, dams) which became the primary reason for the central administration and stationary peasant communities. Scott further explains in his series of lectures  \citep{scott2005civilizations} about how the hill people were incredibly diverse as they spoke hundreds of distinct languages and dialect. Some groups like Akha of Burma and the Hani of China shared similar origins \citep{boonyasaranai2014common} but diverged with time and became culturally distinct. Due to numerous groups and small amount of groups, no group emerged on top. The highlanders refusal to homogenize or fully embrace lowland identities was another form of political defiance. Lowland states typically saw the surrounding hills as lawless peripheries to be exploited or pacified when possible. Valleys would often conduct slave raids to exploit the highlands and loot their resources \citep{walker1999legend}. Captives from hill tribes were used either as slaves or soldiers forcefully into the state societies. Temporary alliances between these two societies were made  but these arrangements were often fragile and broke quickly. Zomia has shown us how there were cultural and political divides between the mountain and plains.

\section{Case Study of Indian States}
In this section we will look at specific states in India where these differences have manifested. These case studies are helpful as a lot of mountain states in India were formed after breaking away from plain societies. The politics around these revolved a lot around the geography of the states.

\subsection{Himachal Pradesh}

 Himachal Pradesh is a state dominant with highlands and the political landscape has shown distinct regional patterns based on differences in geography. The state emerged from the Simla presidency in 1947 with minimal political organization. However, in 1966 state reorganization marked an important turning point and created a distinct divide between \enquote{old Himachal} (old mountainous regions) and \enquote{new Himachal} (merged areas like Kangra and its valley region)\citep{TR_Sharma_1987}. This division was amplified by resource allocation issues particularly water distribution. The older regions became Congress strongholds and were reluctant to share resources with newer areas and fund allocation based on population often disadvantaged the newer regions. The newer agricultural zones became BJP strongholds after the party successfully mobilized local pressure groups. Attempts by parties like BSP and Himachal Vikas Congress to establish themselves have been unsuccessful \citep{chauhan2004bipolar}. This is also visible in the development model proposed by Yashwant Singh Parmar who was the first chief minister of Himachal Pradesh. He presented to the  government that a ``plains oriented model of development" would not work for the state. The  government at that time in their five year plan proposed to  focus on industrialization. However, in Himachal the proposed to focus on rural roads, horticulture, and social services rather than heavy industrialization. To include the remote areas like Lahaul Spiti, Kangra which could be seen as ``Zomia pockets" of Himachal a special force named Task Force on the Development of Tribal Areas and an Expert Committee on Tribal Development were created. Under the program border roads were built to connect these regions and were coerced into the state structure without much resistance \citep{StateEffectiveness2020}. The state has allowed them to retain their cultural autonomy leading to the formation of ``microcosms". Microcosms are defined as ``A small, self-contained world that reflects or represents a larger system or reality". For example the village of Malana in Kullu district also known as the ``hermit village" has its own ancient council and deity governance. Malana has a historically had a great autonomy under their diety Jamlu and even kept the Kullu raja at bay. Malana has argued to be one of the longest surviving Zomia pocket but it is now slowly being brought into the state control with them sending representatives to the Panchayat and when the authorities interrupted their cannabis cultivation \citep{axelby2015hermit}. Thus, in Himachal the trend has been consolidation of state authority in the mountains, balanced by respect for local culture.

\subsection{Manipur}

Manipur presents itself as a very interesting case as it has a clear geographic difference. Inside Manipur is a valley which is surrounded by the hills on all its sides. This is even visible in the lok sabha elections as it has two constituencies Inner and Outer Manipur. After Manipur became a full state in 1972, the hill tribes suddenly found themselves under a state government largely run by valley elites (the Meitei, who are not tribals). They were seen as an extension of the colonial rule which didn't settle with the hill tribes.  In the Naga areas of Manipur’s hills, the Naga National Council (NNC) and later the NSCN (National Socialist Council of Nagalim) propagated the idea of ``Greater Nagaland” (Nagalim) to unite Naga-inhabited hills across Manipur, Nagaland, and Burma. Even the structure of governance is different in the plains and mountains. Imphal valley has a strong presence of state structure with government run schools, police stations etc. However, the hill area is a part of Autonomous District Councils to give them more freedom to take their own decisions. In India, Autonomous District Councils (ADCs) are ``constitutionally recognized bodies established under the Sixth Schedule of the Indian Constitution". . However, Manipur’s ADCs have relatively constrained powers and have often been defunct or boycotted for periods. Additionally, land ownership in the hills is governed by customary law and even people from Imphal valley cannot buy land in hills under the  Manipur Land Revenue and Land Reforms Act (MLR and LR) of 1960. Manipur state government has historically not interfered with tribal land. 

// write more from the og paper

Imphal valley enjoys relatively better infrastructure whereas as the hill areas lag in healthcare, infrastructure and economic opportunities creating more unrest. \citep{lacina2009problem}

Manipur highlands are politically, structurally and government wise different from the valley. This resonates strongly with Zomia as the ``zones of refuge" which have directly avoided a centralised state power.

The government structures and laws to protect the rights of the indigenous tribe shows that it is different from Himachal and even though both of them have small pockets which have resisted state influence, there methods are different. Highlanders in Manipur managed to bargain for more autonomy given there special status in the constitution but it has led to more violence and division in the state.