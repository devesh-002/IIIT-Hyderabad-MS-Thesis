\section{Introduction}
\begin{sloppypar}

This chapter investigates why Duverger’s Law and gender dynamics, fail to manifest uniformly across India, particularly in the mountainous states. While strategic voting is a common explanation for Duvergerian outcomes elsewhere, evidence for its widespread effect in India is contested and inconclusive. Instead, this chapter explores alternative factors rooted in India’s complex social fabric. It focuses on the  influence of geography, specifically the historical and ongoing distinctions between mountain and plain societies. We find examples of these structural differences across India and argue that differing identity formations, unique socio\hyp{}political structures, and distinct gender dynamics along with the idea of Zomia in highland regions, offers a more compelling explanation for varying electoral patterns and social differences. 

\end{sloppypar}
\section{Can Strategic Voting explain the difference?}

Strategic voting is often used to explain Duverger’s law as voters do not vote for their most preferred candidate or party but rather for a less\hyp{}preferred option if they have a better chance of winning. India’s case is different as there is a mixed evidence of strategic voting. 

\vspace{0.3cm}

In the initial studies, Chhibber uses SF ratio to show that strategic voting happens in masses in India. SF ratio is the ratio of votes obtained by the second loser to the first loser. Chhibber argues that if SF ratio is near 0 then there is strategic voting happening at significant levels and vice versa. However, SF ratio is always not a reliable metric as Diwakar mentions 

\vspace{0.3cm}

“SF ratio close to 1 (signifying a non\hyp{}Duvergerian equilibrium) is possible in two situations: first, where the winning party secures a large majority of votes, leaving a very small proportion of votes for the other parties, and, second, where many parties share the votes in a closely fought election.”

\vspace{0.3cm}

Hence, SF ratio is not a reliable metric. While some studies show that Indian voters tend to be more strategic than expressive when their preferred party is unlikely to win \citep{choi2009strategic}, a recent study showed modest evidence for elite collusion explaining the voting patterns in India and also showed that strategic voting is practically absent in India \citep{ziegfeld2021accounts}. A case study on Uttar Pradesh \citep{heath2022so} using survey data showed that there was no evidence for strategic voting and the majority of the people expected their party to win. The study was limited in scope due to its small sample set and evidence from one state only and Uttar Pradesh, which doesn't necessarily follow Duverger's law but, it showed how the metrics used to evaluate strategic voting were weak, and it was first of its kind to use survey data. 

\vspace{0.3cm}

Hence, it can be argued that strategic voting cannot be the case necessarily in these states. India unlike other countries is very diverse with a very complex social fabric with various identities like caste, gender, tribal groups, religion, geography  coming into play and has different reasons for divergence from Duverger's law apart from strategic voting. The case of a single identity being the reason behind explaining Duverger’s law is not a new idea \citep{mayer2013gross} for India. Mayer analyzed parameters like dummy candidates, spoiler candidates, regional parties  and found the ratio of SC/ST i.e. caste based identity being a moderate reason behind the metric of effective parties. 


\section{Role of Identity politics in Mountains}
\begin{sloppypar}
The formation of identities is a continuous process often done as a response to external stimuli. The rise of different type of identities in the mountains and plains can also explain the difference in their electoral and social behaviors. For example, Uttarakhand and Himachal saw a rise of a distinct ``Pahari" identity \citep{mishra2000politics} which was different from the rise of caste politics i.e. caste based identities in the plains. This is despite the fact that caste based cleavages exist in mountains too. For example, almost 50\% of Himachal Pradesh consists of Rajput and Brahman's combined. The formation of Uttarakhand was triggered by opposition for job reservations for OBCs from the plains being applied to hill districts in the 1990s \citep{mishra2000politics}. Uttarakhand had less than 2\% of people as OBCs and were worried that the application of 27\% reservation in hills would lead to plain people taking their jobs. Hence, caste politics acted as a catalyst to trigger the formation of Uttarakhand. However, after Uttarakhand's formation, it was noted that statehood actually accentuated the Pahar Maidan divide within Uttarakhand itself  \citep{mathur2015remote} and Uttarakhand was created in response to the longstanding lack of political representation for the hilly regions within mainstream state politics. The formation of different identities in North East might seem similar to the social cleavages present in the northern plains like caste. However, it is important to note that these identities emerged as a result of geographic differences. The conflict between the hill and plains communities served as a crucial catalyst in the formation of distinct identities, particularly for the hill tribes who sought to differentiate themselves from the perceived cultural and political dominance of the plains populations. For example, due to the demand of Assamese as an official language the hill tribes especially felt threatened \citep{inoue2005integration}.  B. M. Pugh was a Khasi leader who was in favor of creating a state consisting of hill areas only. He wished to combine present day Meghalaya, Nagaland, Arunachal Pradesh, Mizoram together but the motion didn't go through as the central government feared divisions on the basis of religious lines i.e. formation of a Christian state \citep{karlsson2013evading}. However, this did not stop from further divisions from happening as there was soon a rise of  conflicts among these tribes. The most violent and recent are the Kuki (tribe in mountains) and Metei (tribe in plains) as tensions rose in Manipur. Examples of separatist movements due to geographical differences also include the long\hyp{}standing demands for a separate state of Bodoland within Assam  and the autonomy movement of the Karbi community \citep{sarma2017bodoland}. Apart from geographical based divisions, economic disparities also played an important role as hill tribes were often neglected and received minimal support making them feel alienated from the mainland India. Economic neglect was an important factor contributing to the rise of the United Liberation Front of Assam (ULFA) insurgency \citep{chima2023insurgency}. It is interesting to note what caused the rise of a separate strong hill identity across multiple states despite presence of strong ethnic divisions which effected the rest of India. 
    
\end{sloppypar}

The identity of ``Pahari'' or not is also seen in the ENP values of states like Himachal or Uttarakhand in both assembly and Lok sabha elections as they soon converge to two after the formation of state. However, identities is not a complete explanation to these values as formation of identities is a fluid process and there are many external factors effecting. It gives us a broad explanation of what \textit{might} be the reason, however pinpointing the exact reason is difficult to do so.
 



\section{Structural Differences: The Indian Case}
\begin{sloppypar}
    
 In the previous chapter, we observed that mountain societies outperformed their plain counterparts on various social indicators. This section examines gender and political disparities between mountainous and plain regions to evaluate whether mountain societies, as suggested by the Zomia framework, possess distinct advantages.
\end{sloppypar}

\subsection{Political Differences: Plains v/s Mountains}
As Scott argued that plains and mountains are structurally different, social differences like difference between kinship and family structures, economic freedom of women and matrilineal social norms were not the only differences observed between the Zomia and lowland areas. These regions also diverged in their political structures and hierarchies. In the Zomia framework, mountain societies tend to be more decentralized, mobile, and egalitarian, while plain societies are more settled. Plain societies are characterized with more centralized authority and rigid hierarchies \citep{Hammond_2011}.

Political organization in the hills tends to be local and kinship based (village councils, clan elders, tribal chiefs) rather than the hierarchical structures and a top\hyp{}down bureaucracy. Many highland groups formed egalitarian or acephalous (groups with no leader) communities with no permanent chiefs or with only weak leadership roles. Scholars have argued that this was a deliberate choice among these societies. As Scott explains, in general, Zomia's highlanders ``paid neither taxes to monarchs nor regular tithes to a permanent religious establishment”. This is opposite to the valley people who were often forced to pay taxes to the church and the state. Local moneylenders often worsened the situation by employing brute force to collect taxes, which were considered a birthright of the monarchy. For example, the jizya tax was imposed on non\hyp{}Muslims and the development of the zabt system and the dahsala taxation method during the medieval period \citep{moosvi1973production}. Highland societies often favored mobility which led to the development of practicing shifting cultivation or the slash and burn agriculture. These crops are easy to scatter and harvest at different times, making it hard for would\hyp{}be tax collectors to confiscate a single big harvest. The lowland states depended on intensive agriculture (irrigated rice, canals, dams) which became the primary reason for the central administration and stationary peasant communities. Scott further explains in his series of lectures  \citep{scott2005civilizations} about how the hill people were incredibly diverse as they spoke hundreds of distinct languages and dialect. Some groups like Akha of Burma and the Hani of China shared similar origins \citep{boonyasaranai2014common} but diverged with time and became culturally distinct. Due to numerous groups and small amount of groups, no group emerged on top. The highlanders refusal to homogenize or fully embrace lowland identities was another form of political defiance. Lowland states typically saw the surrounding hills as lawless peripheries to be exploited or pacified when possible. Valleys would often conduct slave raids to exploit the highlands and loot their resources \citep{walker1999legend}. Captives from hill tribes were used either as slaves or soldiers forcefully into the state societies. Hence, Zomia has shown us how there were cultural and political divides between the mountain and plains.

From a different perspective, \cite{alam2008becoming} argued that the Western Himalayas could not be studied using social science concepts developed from a study of plains societies. He drew on the example of the Beas, Sutlej and Tons river valleys to describe how the apparatus of the modern state, introduced by colonialism, were what made these historically distinct regions a part of India as it emerged in the 20th century.

Given these political differences, mountain societies often had a difficult relationship with the plain states, even during the post\hyp{}colonial nation\hyp{}state. This has been true not just for the Himalayan states of India, but other post\hyp{}colonial countries like Pakistan, China, Myanmar, etc. Labelled backward and/or `tribal' by the colonial state due to their non\hyp{}state polities, relatively egalitarian political economy and less social hierarchy, mountain societies were also marginalized and `peripheralized' by the post\hyp{}colonial nation\hyp{}state. However, their remote, peripheral locations made this region important in military \hyp{} strategic terms too, particularly after the Sino\hyp{}Indian war of 1962. Border making had started in the 19th century itself as the British tried to fix cartographic lines in the fluid historical geography of the Himalayas and that project remains unfinished, deeply contested and conflict\hyp{}ridden \citep{noorani2010india,guyot2017shadow,acharya2022boundaries}. All the governments of independent India have given special emphasis to `develop' the Himalayan states with special budgetary outlays, constitutional protections and special policies, even if the effort has been uneven and the outcomes even more so \citep{gupta2013allegiance,mathur2013naturalising,mishra2013developing}. The situation is similar in Pakistan where studies of Gilgit-Baltistan and Hunza indicate deep feelings of alienation and isolation among these mountain societies \citep{hussain2015remoteness,ali2019delusional}.  However, it is now accepted that with development and modern infrastructure, the older form of structural distinction between highland and lowland is fading as these regions are now integrated into the nation-states. One of the more insightful of these arguments have been on the function of road building in the Himalayas and how these integrate the mountain societies to the market economies of the plains \citep{murton2013himalayan}. 

% Due to these mountains being porous, it allowed for a 1600+ kilometer border to be open with Myanmar as Indian officials have officially acknowledged the difficulty in closing the same \citep{Bureau_2024}.  


\subsection{Gender Expression: Plains v/s Mountains}
\begin{sloppypar}
    Scott has argued that mountain societies are often more egalitarian and allowed for more gender expression, and even having matrilineal kinship patterns in some cases. This is in contrast with the plain societies which are starkly different due to there patrilineal and male\hyp{}centered lineage systems. This is clearly evident in the \textbf{kinship and family structures of the mountain societies}. For example, in Meghalaya the Khasi and Garo tribes are matrilineal. The youngest daughter inherits the ancestral property, children take their mother's surname, and husbands relocate to their wife's maternal home after marriage  \citep{Allen_2012}. Plain states societies were influenced by Confucian, Hindu or Islamic law and tended to formalize male authority in family and property matters (e.g. requiring sons for inheritance, emphasizing female chastity and patrilocal marriage). These kinship laws have been there for centuries and are still seen in the modern day. It is important to note that this was not limited to kinship laws only. Women's economic roles in mountain communities have generally been as prominent as men's. Women would take charge of household gardens and tend to them. As highland societies were mostly self\hyp{}reliant, women's labor was more visible and indispensable giving them a better economic status. This is not only visible in the Khasi society but, it is also seen in the Lahu people of southwest China (a Tibeto\hyp{}Burman hill group). Lahu men and women share responsibilities in farming, decision\hyp{}making,  and a married couple is considered a single social unit in community affairs \citep{Du_2015}. This egalitarian division of labor has endured among the Lahu, even in the face of sustained pressure from the patriarchal Chinese state over the past two centuries. Plains have often restricted the role of women to childbearing, homes and informal markers which makes there value invisible in the economic markets. Additionally, premarital sex by women might be socially tolerated and women can remarry without stigma. This is seldom allowed in orthodox lowland cultures that emphasized female chastity and one\hyp{}time marriage alliances (often for political or economic gain)
 Men often hold position of responsibilities in public spaces, politics etc. Some highland cultures even allow women to serve as clan heads or spiritual leaders.  For example, numerous ethnic groups in the Southeast Asian Massif have traditions of female `shamans'  who guide towards a spiritual life. However, a few religious positions are often excluded women from leadership (e.g. only men could become Buddhist monks of high rank or Confucian scholars). In valley societies it is often not the case and women had to endure long struggle movements in order for equal rights. In the above\hyp{}mentioned example of matrilineal societies of Khasi hills, men were often allowed to control the  village councils (dorbar) \citep{WashingtonPost_2015,TheGuardian2011}. In plains women's autonomy was often curbed by strict marriage customs, purdah or seclusion practices (in some Hindu and Muslim kingdoms). These restrictions were often reinforced by legal codes that emphasized obedience to fathers and husbands \citep{Papanek_1973,Devi2019}.
\end{sloppypar}

In the previous chapter under NFHS analysis, we saw how the mountain states performed better on indicators which indicated better circumstances for women in mountain states. This has been a historical occurrence as in Uttarkhand and Himachal women have often helped by  working in fields and forests which made them economically productive and helped them learn relevant skills too \citep{gooch2014daughters}. In Plains ``ghoonghat'' or veil was forced to cover women's face. The veil was thought to emphasize on women purity and forced them to stay at home, not being touched. This limited their life and growth and often dependent on their father/husbands making a patriarchal society \citep{chowdhry1993persistence}. Practices like Sati were prevalent in North India especially Rajasthan among the Rajput community. Sati was ``the \textit{forceful} self-immolation of the wife after  untimely death of the husband" \citep{sangari1981sati}. These practices are not present in North East India (although some stigmas are present). The presence of Ima Market which is a market for  women married \textbf{atleast once} in Manipur. In Pochury Nagas women are the head of their families and take over the responsibilities of their husbands. In Himachal and Uttarakhand ``Chipko Movement'' by women allowed the rise of eco-feminism \citep{moore2011eco}. The movement not only protected forests but also empowered women and positioned them at the forefront of environmental activism.  However, women's role in formal decision making is limited in North East. The 2017 protests in Nagaland by all male tribal bodies against implementing a 33\% women's reservation in municipal councils.  In Bihar, UP and Rajasthan have \textbf{atleast} 33\% women's reservation however scholars have noticed that their role is purely ceremonial. Many ``Sarpanch Patis'' are present who actually take the decision on behalf of women \citep{rajasekhar2016women}. Witch hunting has been an issue in a few tribes in Assam where  women, often widows with property or simply vulnerable individuals were targeted and killed over superstitions. It has recently been banned by the government, but it is a recent issue not backed by any customs \citep{mishra2018targeting}.
Northeast communities generally showed less discrimination in child rearing as some studies noted that infant mortality for girls was lower in Meghalaya, Mizoram etc. This shows that they don't have a strong preference for sons unlike plains where girl child mortality has been a big issue \citep{mahanta2013gender}. The  consistent improvement of the factors mentioned for women's development in the previous sections show that it is not just a coincidence but a result of better policy development and culture for women in the Mountains. The lack of stigmas against women in Mountains has allowed them to have an edge over the plains which albeit are now catching up to them slowly. The results from child marriage, literacy show that there is still a lot to catch up as states like Bihar, Jharkhand have almost three times the number of marriages than plains. However, the case of Tripura is an exception because of the customs of child marriage present there. Apart from Tripura, all mountain states consistently show high results. 

\section{Case Study of Indian States}
In this section we will look at specific states in India where these differences have manifested. These case studies are helpful in understanding how the concept of Zomia applies not only at the macro scale but is equally relevant at relatively smaller scales, such as within states. It is important to note that the lower elevation areas in these states are not necessarily plains but, a difference in geography within the state has played a clear part in state politics.

\subsection{Himachal Pradesh}
\begin{sloppypar}

 Himachal Pradesh is a state dominant with highlands and the political landscape has shown distinct regional patterns based on differences in geography. The state was formed from the Simla presidency in 1947 after colonial rule. However, in 1966 state reorganization marked an important turning point and created a distinct divide between \enquote{old Himachal} (old mountainous regions) and \enquote{new Himachal} (merged areas like Kangra and its valley region)\citep{TR_Sharma_1987}. This division was amplified by resource allocation issues particularly water distribution as valleys were hugely dependent on farming. The older regions became Congress strongholds and were reluctant to share resources with newer areas and fund allocation based on population often disadvantaged the newer regions. The newer agricultural zones became BJP strongholds after the party successfully mobilized local pressure groups. Attempts by parties like BSP and Himachal Vikas Congress to establish themselves have been unsuccessful \citep{chauhan2004bipolar}. This is also visible in the development model proposed by Yashwant Singh Parmar who was the first chief minister of Himachal Pradesh. He presented to the central government that a ``plains oriented model of development" would not work for the state. The  central government at that time in their five year plan proposed to  focus on industrialization. However, in Himachal they proposed to focus on rural roads, horticulture, and social services rather than heavy industrialization. To include the remote areas like Lahaul Spiti, Kinnaur which could be seen as ``Zomia pockets" of Himachal a special force named Task Force on the Development of Tribal Areas and an Expert Committee on Tribal Development were created. Under the program border roads were built to connect these regions and were coerced into the state structure without much resistance \citep{StateEffectiveness2020}. The state has allowed them to retain their cultural autonomy leading to the formation of ``microcosms". Microcosms are defined as ``A small, self\hyp{}contained world that reflects or represents a larger system or reality". For example the village of Malana in Kullu district also known as the ``hermit village" has its own ancient council and deity governance. Malana has a historically had a great autonomy under their deity Jamlu, and historically  kept the nearby kingdoms at bay. Malana has argued to be one of the longest surviving Zomia pocket, but it is now slowly being brought into the state control with them sending representatives to the Panchayat and when the authorities interrupted their cannabis cultivation \citep{axelby2015hermit}. Thus, in Himachal the trend has been consolidation of state authority in the mountains, balanced by respect for local culture.

\end{sloppypar}
\subsection{Manipur}

Manipur presents itself as a very interesting case as it has a clear geographic difference. Inside Manipur is a valley which is surrounded by the hills on all its sides. This is even visible in the Lok Sabha elections as it has two constituencies Inner and Outer Manipur. After Manipur became a full state in 1972, the hill tribes suddenly found themselves under a state government largely run by valley elites (the Meitei). They were seen as an extension of the colonial rule which didn't settle with the hill tribes.  In the Naga areas of Manipur's hills, the NSCN (National Socialist Council of Nagalim) spread the idea of ``Greater Nagaland” (Nagalim) to unite Naga tribes in all the hill regions across Manipur, Nagaland, and Burma. Imphal valley has a strong presence of state structure with government run schools, police stations etc. However, the hill area is a part of Autonomous District Councils to give them more freedom to take their own decisions. In India, Autonomous District Councils (ADCs) are ``constitutionally recognized bodies established under the Sixth Schedule of the Indian Constitution". However, Manipur’s ADCs have relatively less powers and have often been seen as center's intervention, hence they have been non-functional and ignored for a long time. Additionally, land ownership in the hills is governed by law and even people from Imphal valley cannot buy land in hills under the  Manipur Land Revenue and Land Reforms Act (MLR and LR) of 1960. Manipur state government has historically not interfered with tribal land. 
Imphal valley enjoys relatively better infrastructure whereas as the hill areas lag in healthcare, infrastructure and economic opportunities creating more unrest. \citep{lacina2009problem}

Scholars have shown that although states like Tripura, Mizoram were successful in creating a single \enquote{Mizo identity}, this experiment failed grandly in Manipur as some presented that the identities created in other states interfered in creating a collective identity in Manipur \citep{hassan2007state}. Manipur became a hot zone of politics due to the proximity of the Naga, Kuki and Meitei tribes which are fighting for independence from the state and are also engaged in an internal tribal conflict. Although the tribes are from similar lineage, the fault lines between them are getting bigger considering the current context. Metei tribes are dominant in the plains and are almost 70\% of the population in Manipur \citep{arora2012politics}. Nagas and Kukis occupy the mountains and there is a huge fight for resources which are already limited. Metei's have consumed a lot of the resources and refuse to share them with Nagas and Kukis which has led to increasing fault lines between them. N There have been lack of efforts by the government to recognize the Nagas and Kukis as the Metei language called as \enquote{Manipuri} has led to a rising feeling of alienation.
\begin{sloppypar}

    Hence, we observe that Manipur's highlands are politically, structurally and government wise different from the valley. This resonates strongly with Zomia as the ``Zones of Refuge" which have directly avoided a centralized state power.
    
     Highlanders in Manipur managed to bargain for more autonomy given there special status in the constitution but, it has led to more violence and division in the state.
    \end{sloppypar}
\section{Conclusion}
\begin{sloppypar}
    
In conclusion, this chapter argues that strategic voting provides an insufficient explanation for the varied adherence to Duverger's Law across India. Instead, deep\hyp{}seated structural and identity\hyp{}based factors, significantly shaped by the mountain\hyp{}plain geographic divide. Studies revealed distinct patterns in identity politics, kinship structures, gender roles, and political organization between highland and lowland areas, drawing parallels with concepts like Zomia. However, it is now accepted that with development and modern infrastructure, the older form of structural distinction between highland and lowland is fading as these regions are now integrated into the nation\hyp{}state. Murton emphasized on the function of road building in the Himalayas and how these integrate the mountain societies to the market economies of the plains \citep{murton2013himalayan} and, these roads have narrowed the gap between the mountain and plains and helped the  state to slowly gain control over these remote areas.
\end{sloppypar}
Case studies of Himachal Pradesh and Manipur illustrated how these geographic and social cultural differences translate into unique political landscapes and electoral outcomes, often diverging from expectations based solely on electoral mechanics. 

