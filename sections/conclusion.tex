\section{Overview}
This thesis investigated the impact of geography on the political and social landscapes of India. Through the lens of electoral competition and party systems using the Laakso and Taagepara Index, this study provides quantitative evidence that there could be a structural discontinuity between mountain and plains polities in India. It aligns with the anthropological, sociological and historical scholarship which has argued for a similar structural distinction between mountain and plains societies by indicating that even electoral behaviour in the recent past arranges itself into this geographical divergence. Guided by  theoretical frameworks of political geography, electoral studies, and particularly James C. Scott's concept of "Zomia" as we test the hypothesis using a mixed methods approach.

\subsection{The Divergent Paths of Political Competition: Duverger's Law in Geographic Context}

Our study has revealed that mountain states of India have consistently fewer effective parties than the plains states and show a clear trend of converging towards the two party competition system. While the negative slopes in ENP trends vary among the mountain states, the general pattern is that overall there is a decrease in ENP values. In contrast, plains states generally exhibit higher ENP values with positive slopes due to increasing party system fragmentation. This is particularly seen in state assembly elections for plains where the ENP values have steadily increased over the decades and frequently surpassing the moderate threshold of 2.5 and often exceeding the higher threshold of 3.0. We have found a statistically significant difference between mountain and plains states, and a post-hoc analysis using Cohen’s d score confirms that the results are not coincidental. The examination of gender dynamics through NFHS data provided a strong evidence for structural differences between the two regions. We choose indicators like child marriage, female literact, contraceptive use and breastfeeding practices to test for the difference. We find that the mountain states consistently outperformed the plain states. Using the rankings from the indicators, we make a composite ranking and find that the mountain states dominate the composite rankings derived from these indicators. States like Himachal Pradesh, Mizoram, Sikkim, and Nagaland frequently appeared among the best performers, compared to plain states like Bihar, Rajasthan, Uttar Pradesh, and Jharkhand, which consistently occupied the lower ranks despite nationwide improvements over time. These findings align with Scott's arguments suggesting that highland societies are historically less penetrated by patriarchal state structures and women's labor is highly visible and valued which tend to foster greater gender equality and autonomy for women. The only difference was in contraceptive use which can be explained by factors related to remoteness and accessibility of modern healthcare services and family in remote mountains. 

\subsection{Synthesizing the Divide: Geography, Identity, and the State
}

Given that we are using a mixed methods approach, we tried to back our quantitative findings by doing a qualitative review. We found that strategic voting has found mixed evidence in literature in Indian societies. On one hand \cite{choi2009strategic} suggests that strtaegic voting might be present, recent studies by \cite{ziegfeld2021accounts} using more data  suggests against this idea. Hence, it is difficult to take this idea in account. We find that the observed differences are not merely a fluke but appear rooted in distinct historical trajectories, social structures, political economies, and state society relations characteristic of highland and lowland zones.  We also look at the process of identity formation. The electoral politics in the plains region has tended to develop its political forces based on caste and religious identity whereas the mountains have produced numerous identities that emerged through opposition against both plain dominant beliefs. This often results in the formation of broader regional (``Pahari'') or ethnic identities (Naga, Kuki, Mizo, etc). We observed examples of how the kinship systems, markets, culture allowed women more freedom in the mountain areas like the examples of matrilineal societies in the Khasi tribes is an important example. Whereas oppresive policies like the ``purdah'' or ``sati'' in the plains points towards the historic oppression which women had to face.  The post colonial Indian state struggled to integrate peripheral areas through methods that included both forced strategies and dialogue processes as well as administrative autonomy bodies and development projects. We also show that the idea of differences is not only prevalant at a national level but also the state level. These case studies reveal separate outcomes from this integration process because Himachal Pradesh experienced stability while Manipur struggled with continuous conflict yet show both regions share the persistent importance of their hill valley geographical and political division.

\section{Limitations}
While this study offers significant insights, it is essential to acknowledge its limitations. The exclusion of Jammu and Kashmir, due to its complex geopolitical situation and data availability issues, means the analysis does not cover the entirety of India's mountainous north. The anthropological and historical scholarship suggests that this is due to a structural difference between the plains and the mountain societies, and the findings of our study tend to support this hypothesis, there may be a need to pause before accepting this claim fully. This is so because we have not tested out other possible correlations and causal links. In this study we have based ourselves on the fact that mountain geographies create a historical sociology which is distinct. Our study here suggests that parallel to this, there is a distinct political culture in the mountain states which stands them apart from their plain neighbours. Even when they were part of the plains state (like Uttarakhand was of Uttar Pradesh) the mountain regions displayed a different political – electoral behaviour. But what would the results be if we controlled for literacy and education, or for demographic features? Can there perhaps be other aspects which we are not even cognisant to?  What explains the variations within the mountain states? What other factors may influence political culture and behaviour in mountain states other than geography dependent on historical sociology?  

\section{Future Work}

These limitations pave the way for future research. We can extend the time series analysis for electoral data. The analysis of political and social outcomes can be enhanced when using GIS to examine specific territory types at constituency and village resolutions. Field based qualitative research involving ethnography alongside surveys with participants from mountains and plains can be used to discover how political dynamics and identity development work within the observed system. This research should focus on what drives voter behaviors and how women live their lives alongside network functions between politicians and the particular aspects that shape cultural identities. Geographical variables need to be accounted while doing the statistical analysis in order to form a causality between terrain and politics. Studies involving additional countries that have major upland versus lowland divisions (specifically in Southeast Asia and Latin America) would strengthen the application of Zomia theory and the results from this investigation. 

\vspace{0.3cm}

This thesis explains the necessity of incorporating geographical perspectives into the study of India. The specific features and difficulties faced by India's mountain populations need specialized governmental response while the nation follows its development and modernization trajectory because these communities maintained their distinct identity through their  connection to both land and state. Hills and valleys in India work as vital elements which actively determine how identities emerge along with controlling power distribution and determining government capabilities and human possibilities for thriving. Research reveals India's complete understanding requires studying both the active plains region and hearing the particular sounds emanating from mountain areas.  The investigation of particular state policies including infrastructure development alongside resource distribution and governance changes must constitute the focus of scientific inquiry for understanding regional political and social transformations.