\begin{sloppypar}

\section{Overview}
This thesis investigated the impact of geography on the political and social landscapes of India. Through the lens of electoral competition and party systems using the Laakso and Taagepara Index, this study provides quantitative evidence that there could be a structural difference between mountain and plains polities in India. It aligns with the anthropological, sociological and historical scholarship which has argued for a similar structural distinction between mountain and plains societies by indicating that even electoral behavior in the recent past arranges itself into this geographical divergence. Guided by  theoretical frameworks of political geography, electoral studies, and particularly James C. Scott's concept of ``Zomia" as we test the hypothesis using a quantitative approach.


Our study has revealed that mountain states of India have consistently fewer effective parties than the plains states and show a clear trend of converging towards the two party competition system. While the negative slopes in ENP trends vary among the mountain states, the general pattern is that overall there is a decrease in ENP values. In contrast, plains states exhibit higher ENP values with positive slopes due to increasing party system fragmentation. This is particularly seen in state assembly elections for plains where the ENP values have steadily increased over the decades and frequently surpassing the moderate threshold of 2.5 and often exceeding the higher threshold of 3.0. We have found a statistically significant difference between mountain and plains states, and a post-hoc analysis using Cohen’s d score confirms that the results are not coincidental. The examination of gender dynamics through NFHS data provided a strong evidence for structural differences between the two regions. We choose indicators like child marriage, female literacy, contraceptive use and breastfeeding practices to test for the difference.  Using the rankings from the indicators, we make a composite ranking and find that the mountain states dominate the composite rankings derived from these indicators. States like Himachal Pradesh, Mizoram, Sikkim, and Nagaland frequently appeared among the best performers, compared to plain states like Bihar, Rajasthan, Uttar Pradesh, and Jharkhand, which consistently occupied the lower ranks despite nationwide improvements over time. We find that the mountain states consistently outperformed the plain states. 

In our study we have tried to back our quantitative findings by conducting an in depth qualitative review. We found that strategic voting has found mixed evidence in literature in Indian societies. On one hand \cite{choi2009strategic} suggests that strategic voting might be present, recent studies by \cite{ziegfeld2021accounts} using more survey data  suggests against this idea. Hence, it is difficult to take only strategic voting in account. We find that the observed differences are not merely a fluke but appear rooted in  historical trajectories, social structures, political economies, and state society relations characteristic of highland and lowland zones.  We also look at the process of identity formation. The electoral politics in the plain region has tended to develop its political forces based on caste and religious identity whereas the mountains have produced numerous identities that emerged through opposition against both plain dominant beliefs. This often results in the formation of broader regional (``Pahari'') or ethnic identities (Naga, Kuki, Mizo, etc). We observed examples of how the kinship systems, markets, culture allowed women more freedom in the mountain areas like the examples of matrilineal societies in the Khasi tribes is an important example. Whereas oppressive policies like the ``purdah'' or ``sati'' in the plains points towards the historic oppression which women had to face. However, it is difficult to attribute the difference solely to identities, as they are shaped by multiple intersecting factors.  These findings align with Scott's arguments of  ``Zomia", suggesting that highland societies are historically less penetrated by patriarchal state structures and women's labor is highly visible and valued which tend to foster greater gender equality and autonomy for women. The only difference was in contraceptive use which can be explained by factors related to remoteness and accessibility of modern healthcare services and family in remote mountains. We also show that the geographical differences  not only exist at a national level but also the state level. These case studies reveal separate outcomes from this integration process because Himachal Pradesh experienced stability while Manipur struggled with continuous conflict yet show both regions share the persistent importance of their hill valley geographical and political division.

\section{Limitations and Future Work}
This study avoids testing explicitly for the potential influence of variables such as literacy rates as well as fertility levels, per capita income and other demographic indicators. The described elements play major roles in molding political conduct and account for some regional dissimilarities seen in party system structures. We should emphasize that our main purpose was to develop an empirical validation based on Duverger's Law. The results demonstrated that the election systems of India's mountain states show two-party system characteristics which is different from the plains. \textbf{Our assumptions point toward the possibility that different political cultures within mountain societies cause the observed results, yet we avoid definitive causal conclusions.} The research results act as preliminary evidence to support predominantly theoretical interpretations within state-making and mountain society scholarship. But what would the results be if we controlled for literacy and education, or for demographic features? Can there perhaps be other aspects which we are not even cognizant to?  What explains the variations within the mountain states? What other factors may influence political culture and behavior in mountain states other than geography dependent on historical sociology?  

Despite these unanswered questions, we can state that the findings of our study contribute to our understanding of how geography intersects with political institutions and suggests that theories of party system development, perhaps even theories of federalism, need to account for historical geography. Our study also opens up paths for other studies which not only explore the unanswered questions listed above, but also explore whether similar patterns exist in other mountain societies globally and investigate specific mechanisms through which historical geography influences party system development. It provides a fillip to further comparative studies both within nation-states and among them. Extending these studies to the quarter century between 1952 and 1977  could also qualify our findings. We also see the possibility of going more granular with quantitative studies of electoral behavior and political culture at municipal and panchayat levels, and comparing them to state and national levels. 

Given all these caveats, limitations and untapped possibilities that we have listed out, we still feel that our study has put forward a strong case for extending the scope of Zomia from a purely historical and anthropological concept to one which can also help explain contemporary politics. In fact, Scott argues that Zomia, as a distinct historical geography, starts to fade out in the second half of the 20th century. The present study suggests that this may not be so, or at the very least, the fading out is a far slower process than the votaries of Zomia have argued for.  

\vspace{0.3cm}

This thesis explains the necessity of incorporating geographical perspectives into the study of India. The specific features and difficulties faced by India's mountain populations need specialized governmental response while the nation follows its development and modernization trajectory because these communities maintained their distinct identity through their  connection to both land and state. Hills and valleys in India work as vital elements which actively determine how identities emerge along with controlling power distribution and determining government capabilities and human possibilities for thriving. India's complete understanding requires studying both the active plains region and hearing the particular sounds emanating from mountain areas.  The investigation of particular state policies including infrastructure development alongside resource distribution and governance changes must constitute the focus of scientific inquiry for understanding regional political and social transformations.

\end{sloppypar}