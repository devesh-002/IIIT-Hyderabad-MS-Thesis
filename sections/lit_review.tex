\section{Introduction}
In this chapter, we conduct a review of studies pertaining to Indian politics and cover a brief history of Indian politics. This chapter provides an overview of the evolution of Indian politics, focusing on the historical context and the factors that have influenced its trajectory over time. India started out with Congress as the single largest party \citep{kothari1967india} but it soon fragmented to give rise to smaller parties due to India's diversity with people having different identities from various faiths, castes, creed etc. We conduct a review of what were these identities and how they manifested differently in plains and mountains not only in India but across the world.  However it is difficult to operationalise these identities quantitatively, political scholars have attempted to do it so using the Duverger's law. The law has had a profound impact around the world predicting the rise and fall of parties too. In this study, we also use a similar approach to quantify and conduct a review over Duverger's law in India and across the world \citep{duverger1954political}. Another well known explanation for differences between mountain and state is the concept of Zomia, a phenomenon that is not confined to India but is observed worldwide. In the literature review, we explore the concept of Zomia in depth and examine how similar contemporary theories have emerged in different parts of the world. 

\section{Electoral Politics of India}
\subsection{Post Independence Era}
 From 1952 to 1967, the Indian political landscape was largely dominated by the Congress party. This was due to Congress being the face of Indian struggle against the British rule \citep{shastri1991nehru}. Congress established a political hegemony as \cite{kothari1967india} pointed out it being an ``umbrella organization". In this system Congress party formed a  coalition featuring representatives from all castes, religions and ethnicities to account for the diverse interests in India \citep{anand2015downfall}. It formed a careful system of checks and balances to account for these groups and resolve disagreements. \cite{kothari1967india} also called it as a ``party of consensus" as it tried to emulate the diversity of India in the party so that the internal factionalism within the Congress served as a mechanism for balancing power and addressing various societal demands. However, some factions felt that there demands were not being listened and felt alienated from the decision process. This lead to rise of smaller groups with distinct identities unlike Congress who advocated for a collective nation building \citep{shastri2003continuity}.  Congress’s inclusion of various sectors was symbolic and it
was headed by elite leaders only. The Congress system did help democratic ideas grow and let society try out changes safely but it wasn’t good enough for full-blown competition
in politics with big social changes \citep{shastri2009electoral}. Scholars classified it as a system of uni polar hegemony where deep social changes are not possible. As a result, congress faced its biggest challenge from Lok Dal in the 1960s \citep{desouza2006india}. Post independence, India was divided in two parts due to partition which laid the foundation of India's divide on the basis of religion. This, along with rising tensions between different castes led to formation of new ``identities" and rise of identity politics in India. 
\subsection{Rise of Identity Politics in the Northern Plains}
Stanford Encyclopedia of Philosophy \citep{Heyes_2024} defines identity politics as \begin{quote}
     A tendency for people of a particular religion, ethnic group, social background, etc., to form exclusive political alliances, moving away from traditional broad-based party politics.
 \end{quote}
In India, identities were formed on the basis of caste, religion, language, ethnicity etc. These identities started gaining momentum in 1960s which led to the State Reorganization Commission which divided Punjab in Haryana (a Hindi-speaking, Hindu-majority state) and transferred a few areas to Himachal Pradesh \citep{Punjab_1966_reorg}. It is interesting to note Congress’s support base. In 1980, the Congress won 50 of the 79 reserved Scheduled Caste constituencies and 29 of the 37 Scheduled Tribe constituencies but it also carried the prosperous sections of New Delhi. Congress heterogenous support group gave it power in various states but also made it fragile at the same time. It was difficult to maintain such a support group in different sects of society and with each
iteration of elections and rise of state parties, Congress kept losing its base. The rise of Janata party in the 1970s and introduction of Mandal commission led to rise of a ``Market, Mandir and Mandal" politics in India \citep{yadav1999electoral}. The differing caste politics forced parties to adapt their strategies regionally and social engineering became key. For example in UP in recent elections the BJP’s candidate selection  included many OBCs (including Non-Yadav OBC groups like Kurmi, Lodh, Jat, Gujjar) and Dalits, alongside upper castes \citep{jaffrelot2012castes}. In Bihar BSP despite its Dalit core base, started wooing Brahmins since the 2000s (“Brahmin-Dalit bhaichara” committees) to expand its appeal \citep{ankit2018caste}. These identities were not limited to caste only. In Punjab, religious identity (closely tied with linguistic and regional identity) has been central but took a different trajectory. Even after the state re-organization commission,  unresolved issues like the status of Chandigarh and sharing of river waters increased tensions \citep{padhiari2008inter}. This led to rise of separatist movement in the 1980s and a separate ``Sikh" identity which is still a part of politics led to rise of communalism in Punjab \citep{gupta1985communalising}. These identities often mixed with each other too. This was noticed in Bihar during 1990s after the implementation of Mandal Commission which caused a huge backlash from the upper castes. This coincided with the rise of Ram Janambhoomi movement too and was termed as ``Mandal vs Kamandal" politics by analysts \citep{roy2024politics}. 

\subsection{Politics in Northern Mountains of India}

Most mountain states in India were formed after separating from plain states and were slowly incorporated in India. A lot of North-Eastern mountain states were given a state status under the \cite{North_eastern_reorg_1971}.  Politics in Himachal Pradesh is dominated by upper castes as  Rajputs and Brahmins together constituting about 50\% of the population. However the politics in both Himachal and Uttarakhand does not revolve around caste. Instead, it revolves around a regional distinct identity i.e. ``pahari" identity \citep{mishra2000politics}. However it doesn't mean that caste based politics is absent in Northern Mountains. The formation of Uttarakhand was triggered by opposition to job reservations for OBCs from the plains being applied to hill districts in the 1990s \citep{mishra2000politics}. Uttarakhand had less than 2\% of people as OBCs and were worried that the application of 27\% reservation in hills would lead to plain people taking there jobs. Hence, caste acted as a catalyst to trigger the formation of Uttarakhand.  Sikkim transitioned from a monarchy to become the  state of India after a referendum held on April 14, 1975 \citep{code1979volume}. Mountain states were often given special status like the Autonomous district councils  designed to provide self-governance to preserve and promote the cultural and social practices of indigenous communities \citep{pautunthang2024india}. A lot of tribes in North East were given SC/ST status too. The Assam province inherited from the British initially included much of the region (except Manipur, Tripura, Sikkim). However, tensions emerged as Assam advocated for Assamese to be its sole state language under the Assam Official Language Act of 1960. Soon, calls of new separate hill districts began and hill leaders started to rally massive support under them \citep{inoue2005integration}. The formation of All Party Hill Leaders Conference legitimized the movement and the struggle officially started. Nagaland was the first state to be formed in 1962 after a decade of violent insurgency. However, scholars have presented that formation of Naga state was due to India's war with China.  A section of Naga leaders initially lobbied for joining the Union of Burma (which had its own Naga tribes and a more federal arrangement at the time), though this did not materialize \citep{Wouters_2023}. Northeast was viewed as a strategic frontier where local unrest had to be quelled swiftly \citep{johari1975creation}. In 1972, Meghalaya was formed as a response to the movement for Garo and Khasi hills. Manipur and Tripura, which had both been princely states that merged into India were also given statehood in 1972. However, Manipur saw violent uprisings due to various reasons which we will study later. Arunachal Pradesh (formerly NEFA) was awarded full statehood in 1987. It followed a different trajectory as it was under Elvin Verrier where he advocated for isolationist policies and slow integration of NEFA in India while respecting tribal rights \citep{verrier_elvin_2008}. However after the 1962 war, the Indian state began increasing its influence in the region due to its proximity with China claiming it to be a part of South Tibet. Thus, Arunachal’s statehood (1987) was as much an international statement rather a response to local demand ( the movement for statehood there was minimal compared to other states).

\section{Duverger's law}
\subsection{Definition}
Maurice Duverger, a prominent French political scientist introduced a principle in the mid-20th century that has since become foundational in the study of electoral systems. Duverger's law states that electoral systems that follow a single-member plurality system (SMPS) such as India where the winner takes all tend to result in the dominance of two major political parties. Popular examples for this are the USA and UK elections, where this has been observed. However, this is not a firm law as it does not hold in India and Canada.
Two primary mechanisms have been identified through which electoral systems influence party structures in relation to Duvergers law:
 \begin{enumerate}
     \item \textbf{Mechanical effect:}  In a single member plurality system, the party receiving the most votes wins. This means that the smaller parties often struggle to secure representation. This process causes over representation of larger parties and under representation of smaller parties leading to a concentration of political competition between two dominant parties. Often, Duverger's law has been used to study national level competitions but the essence of this law has been often identified on district level \citep{cox1997making,GALLAGHER199133,lijphart1994,rae1971political}. Increase in competition between multiple parties at a district level indicates a higher competition at national level too.
     \item \textbf{Psychological effect:} This is a direct response to the mechanical effect by the voters and political elites. For example, knowing that smaller parties have little chance of winning, voters avoid wasting their votes on them, instead opting for one of the major contenders. Similarly, political elites may choose not to enter the race under unfavorable conditions or may form coalitions to enhance their viability.
 \end{enumerate}
\subsection{Duverger's law around the world}
Duverger's law has been a part of various debates around the world and has found it to be applicable in the USA (Republicans vs Democrats) and UK (Conservatives v/s Labour). In UK, smaller parties like Liberal Democrats Party often receive a decent vote share but almost no parties. Originally, it was presented only as a theory but with time many mathematical proofs have emerged to prove it. \cite{palfrey1989mathematical} presented a mathematical proof of Duverger’s Law under strategic voting conditions using game theoretic models. \cite{cox1997making} presented a study where he offered a general theory and proof of Duverger's law. He presented an $M+1$ rule. The $M+1$ rule argued that in a district with $M$ representatives and system where person with most votes wins with no propositional representations, no more than $M+1$ candidates would exist. In case of Duverger's law $M=1$, hence it predicts at most $2$ parties. Duverger's law has often been studied as static i.e. the equilibrium has remained for a long time. Studies by \cite{forand2015dynamic} showed how  countries move toward or away from the Duvergerian equilibrium over time.  They show that strategic behavior can lead to convergence toward two-party competition over time if any unexpected shocks don't happen. This is specially important in the context of the thesis as we explore whether states converge to Duverger's law over time slowly. 

\vspace{0.3cm}

Duverger's law has been noted in countries where the voting system has changed providing a natural experiment. In New Zealand, it had a two-party system (National vs. Labour) under FPTP (First past the post) for two decades. After 1996 it switched to a mixed-member proportional (MMP) system. This resulted in smaller parties gaining representation proportionately to their votes and New Zealand became a multi party system \citep{Eberhard_2017}. The opposite happened in Italy where they switched from a PR system to  adopting a largely plurality-based mixed system in the 1990s. This led to there party system changed from being a highly fragmented multi-party system to a dual party competition \citep{reed2001duverger}. Essentially, parties merged or formed pre-electoral alliances to avoid splitting the vote in the districts. Similarly,  Japan shifted from a single non transferable vote (SNTV) system (multi-member districts) to a mixed system with single member districts in 1994. Under SNTV, Japan had one party dominance (the LDP) but also multiple smaller parties and intra party factional competition. Under the new system, the party system reorganized into roughly two major blocs (LDP vs. opposition) in many districts \citep{reed2007duverger}. 

\vspace{0.3cm}

However, there have been critiques of the Duverger's law. It doesn't follow in India and Canada \citep{gaines1999duverger}. The case for Indian exceptionalism will be elaborated later. Even though the law is studied on a national level, Duverger himself presented that the law is best understood at a district level \citep{diwakar2007duverger}. There have been limits of strategic voting (the psychological effect) in explaining the Duverger's law. Different reasons have been found to do so. In some cases, people vote for there preferred party to express there protest \citep{ziegfeld2021accounts}. Coordination failures (where supporters of an alternative can’t agree on which major party to back) or protest voting can lead to more than two significant parties even under FPTP \citep{singer2013duverger}. Duverger's law has resulted in limiting voter's choice marginalizing minority voices, and polarizing politics into two ideologies. Mathematically it has been formulated that Duverger's law often leads to parties having the same ideology or completely polarizing opposing ideologies \citep{fey2007duverger}. However, in practice it has been seen that parties often converge to similar ideologies because polarizing ideologies often lead to rise of a median party. Hence, it leaves very little practical choice for the voters.

\subsection{Duverger's law in India}

First attempts to study Duverger's law in India was done by \cite{riker1982two,riker1976number} where he explained the Congress Umbrella in India and postulated that India follows Duverger's law. However, India's divergence from Duverger's law was first presented by \cite{lijphart1994} as he argued that Congress was in a special position due to them being a figurehead of India's independence movement. He argued that India's vast social diversity, including various ethnic, linguistic, and religious groups would lead to social cleavages and predicted the rise of smaller regional parties. \cite{taagepera1989seats,sridharan1997duverger} also presented the same rational as above and predicted a rise of local parties. The first statistical analysis on India specifically is done by \cite{chhibber1998party} who presented in an extended analysis that India follows the Duverger's law and reported the India's ENP at the time to be $2.5$. In an extended study, they first studied the Indian districts and presented that India's districts followed Duverger's law \citep{chhibber2009formation}. Diwakar and Chhibber also used Lok Sabha and assembly constituencies respectively to study the trends of Duverger’s law. 
India as a notable exception to Duverger's law was later extensively documented in the literature \citep{diwakar2007duverger, diwakar2010party, mayer2013gross,carneggie_duverger}. While Diwakar's analysis primarily focused on district level electoral competition through Lok Sabha constituencies, subsequent research has expanded to include Assembly constituencies as well. Studies on assembly constituencies attributed deviations from Duverger's law to India's federal structure \citep{chhibber2006duvergerian}. However, most existing studies have approached this analysis to analyse India as a whole with only limited examination of state level variations. Mayer made important contributions by analyzing plains states but, there remains a significant gap in understanding how Duverger's law operates in India's states and variations between regions.  Peter Mayer’s work extends this literature by focusing on ENP values across 15 of India’s most populous states. However, his analysis largely omits the Himalayan and other mountain states. Mayer  investigates qualitative factors such as dummy candidates, party fission, and the presence of regional parties, but finds limited explanatory power in these variables. 

This thesis builds upon this body of work by introducing a new explanatory dimension: geographical structure. We argue that the divergence in ENP trends between India’s mountain and plain states may be rooted in long-standing geographical and historical differences. This  idea has been explored in qualitative anthropological and historical literature through the concept of Zomia. Although the effects of modernization and state integration are gradually narrowing these historical divides, our findings suggest that remnants of these structural differences continue to shape electoral outcomes.


\section{Mountains Different from Plains}
\subsection{World wide}
\label{mountainww}
% Geography has had a major impact on political attitudes and behaviors, shaping communities and lives through its pervasive influence. Beyond commerce, this geographical advantage helped the British win wars in both medieval and modern eras, establishing the nation as a superpower \citep{young1987geography}. These don't need to be nationwide as in Chicago the demographic composition of passengers on Chicago's Red Line train visibly shifts along racial lines as it travels from the northern to the southern neighborhoods.  These physical separations create ``psychological distance" amplifies existing tensions and adds to biases leading to larger movements for more representation \citep{kasperson1965toward}. Similar geographical influences were observed during the Cold War, when the proximity of Cuba and Nicaragua to the United States posed significant threats due to the spread of communism in America's ``backyard". Hence, we can observe how geography has influenced the macro processes of countries or unions building or destroying the world.  The isolation of the Soviet Union and Japan from their counterparts contributed to their respective downfalls. The influence of geography on politics extends beyond international relations and can be observed in different ways.

% \vspace{0.3cm}

Geography has worked as an escape zone for various people in the past who wanted to escape the control of monarchies or colonial power. These geographical divides have often led to rise of resistant movements across the world. In the Americas, Maroon communities consisted of escaped slaves often living in hard to reach areas like mountains or dense forests resisting colonial control. Their descendants emerged as a form of resistance to slavery \citep{price2020rainforest}. They created resilient communities in inaccessible regions such as mountains, swamps, and dense forests. Jamaican Maroons forced the British to sign treaties and Suriname Maroons persisted despite state pressure \citep{Cultural_Survival_Bilby}. Geography helped the  black, Indigenous, queer and poor people to escape the dominant system where they were not accepted. They were called ``undercommons" and used cracks in societies like universities to escape the state control \citep{harney2013undercommons}. Anthropologist studies have shown how remote communities have tried to avoid centralized authority and are acephalous (headless) in nature \citep{graeber2004fragments}. Examples like Tiv of Nigeria and the Piaroa of Venezuela show how they avoided power in one hand. Tsimihety of Madagascar illustrates how they evaded both monarchy and colonial rule through strategies of withdrawal and dispersal. Authors have argued that instead of being backward or primitive, these societies are stateless by choice. Using technology to there advantage along with legal and international avenues many small communities have exercised there right to remain in isolation \citep{bodley2012anthropology,bodley2014victims}. Bodley develops the idea of ``adaptive governments" which are based on consensus systems to defend against larger corporations. Vandana Shiva presents how small scale farming systems allow communities to resist corporate and state control over food systems \citep{hrynkow2018}. She argues that ``seed sovereignty" is very important for farmers as it allows them to be independent and self reliant. These tribes also practiced such farming practices to evade state control. The above examples clearly show that geography in terms of forests, swamps, mountains have clearly played an important role for people to run away from state control. Southeast Asia is home to some of the world's tallest mountains like the Himalayas and is also the region where ``Zomia" is located.

\subsection{Indian Case}
The division between mountain and plain societies has been evident in the Indian subcontinent since the colonial era. The British Empire in India governed the mountainous Northwest Frontier (today’s Pakistan-Afghanistan border) through indirect means. British enforced the Frontier Crimes Regulations via local Pashtun maliks rather than imposing regular law. This meant more local autonomy for the Pashtun chieftains and acknowledged the difficulty of Britishers in gaining control \citep{ali2013indigenous}. This colonial arrangement remained in independent Pakistan as the Federally Administered Tribal Areas (FATA). This was a longstanding issue and  Pakistan finally merged FATA into Khyber Pakhtunkhwa in 2018 after years of struggle and militancy \citep{horgan2008leaving}. A similar issue can be observed in the Jammu \& Kashmir state of India where a special status under the Article 370 was granted to the state which was abruptly abolished in 2019. States have been forced to give away there autonomy and give local exemptions to incorporate these regions in the state. Article 371-A in India, gives Nagaland state control over its customary laws and land rights. Sikkim has been given special exemption from income tax laws as Sikkim operated under its own tax laws before it was merged into India in 1975. These exemptions were preserved under Article 371F of the Indian Constitution. Similarly, autonomous constituencies are present in  Assam, Manipur and Ladakh to empower indigenous communities and ensure self-governance \citep{sixthschedule}. These councils can make laws on subjects such as land use, forest management, agriculture and village administration. They also have the authority to establish courts for cases involving tribal members, provided the sentences do not exceed five years of imprisonment. Apart from the provisions in the constitution there have been electoral difference in the voting patterns as explained in the results in the previous chapter. 

\section{Zomia}
 Zomia is a term coined in 2002 by Dutch scholar Willem van Schendel to describe a vast highland region on the fringes of South and Southeast Asia. The name derives from Zomi, meaning “highlander” in local Tibeto-Burman languages \citep{van2005geographies}. 
 The evidence of mountain societies being structurally different from plains was seen in India where Van Schendel \citep{van2005geographies} presents the idea of Zomia in which he argues that the borders drawn between major states are arbitrary and were without consideration of the social/cultural boundaries. 
Scott \citep{jamesscott} elaborates on this and extends the existence of the Zomia framework, a stateless society which was the last escape zone and resisted incorporation into the power of state. In modern day, the Zomia region consists of the Mountains in North, North east of India, Tibet and mountainous regions of South east Asia (Himalayan mastiff). Initially the central Himalayas were not a part of this framework but studies  show that the Zomia framework can be used to explain the Central Himalayas \citep{shneiderman2010central}. These remote areas in the Himalayan mastiff were often used to exile unwanted people due to religious and ethnic conflicts but Scott argues that it was actually the opposite. The majority people in these societies deliberately left the state in order to escape it and do trade without any restrictions and escape the crutches of hierarchical divisions and feudal governments to form more egalitarian governments which gave more freedom to women too. These areas were important passes and present on international routes and hence could be easily controlled. Due to their importance, attempts were made in history by kingdoms to incorporate them into states but mostly backfired due to the extreme remote nature of these districts. Often these areas were ignored by scholars due to the remoteness and lack of documented history especially in the Chinese side of the Himalayas which has very strict rules for journalists and data collection. All these restrictions are increased due to the lack of knowledge about the language making it a difficult but important region to study. Although labeled backward/tribal by the state due to their limited history, Scott argues that they have deliberately avoided writing and not have written records. The oral history of these areas becomes an important aspect of study for us. He argues that such states tend to be politically different from the mainland and are egalitarian and free of the crutches of hierarchy like caste which is prevalent in mainlands of India. Although remote, this region has been a very important strategic location due to India and China in close contest against each other in this region who are trying to win over the locals to gain control over important geographical points  and more natural resources like Brahmaputra river and its massive basin. The idea to control eastern himalayas was conceived by the British but due to remoteness, the eastern Himalayas were the last regions not to be captured and trade routes were established to China through current day Assam along the Inner Line. The Inner Line was established in the Eastern Himalayan frontier to regulate movement and interaction between the plains of Assam and the tribal areas of the hills. Post independence, the NEFA (North-East Frontier Agency), present day Arunachal Pradesh was a contested territory between India and China. Both the states were trying to appease the locals and establish control \citep{guyot2017shadow}. This control is often achieved by building \enquote{spheres of influence} around important nodes and grow them \citep{Farrelly_2013b}. In India, Miao in Arunachal Pradesh is an important node of control for the state to gain access to the otherwise remote region. Even though remote there are tools like all season roads, circuit houses and especially schools and colonies of government officials used by government to spread its influence. District collectors are often appointed from the state of Arunachal or Assam to woo the locals which provides the much needed local support and legitimacy to the Indian government.  

\vspace{0.3cm}

 \enquote{The region has never been united politically, neither as an empire nor as a space shared among a few feuding kingdoms, nor even as a zone with harmonized political systems. Forms of distinct customary political organizations, chiefly lineage-based versus \enquote{feudal} unlike plains where feudal systems developed and were controlled by a small elite. They subjugated egalitarian groups in their orbit, but never united, and were never totally integrated into surrounding polities.} - \citep{michaud2017s}

\vspace{0.3cm}

 Even though the existence of Zomia shows political distinctiveness from plains, Scott presents that this was till 1950 only. After that with technological innovations and Zomia becoming a contested area as it became part of borderlands, states quickly developed to incorporate them into their structure and the Zomia came to an end. However, with development of inter-border roads scholars have presented that this might have reopened the debate of Zomia as the state built infrastructure facilitates the movement between these areas opening new opportunities for these markets and areas \citep{murton2013himalayan}. 

 \section{Conclusion}

 In this chapter we looked at the existing literature and found that Congress decline led to rise of much smaller parties which targeted focus groups based on caste, religion or language. However, these differences were \textbf{initally} only seen in the Northern Plains. The mountain states presented a separate ``pahari'' identity based on the geographical differences. The formation of separate on the basis of geographical differences is not endemic to India. It has been seen throughout the world in the form of resistance movements in the USA as Maroon communities \citep{price2020rainforest}, Tsimihety of Madagascar, Nigeria , Venezuela etc. For India, this is a gap which has been addressed by various scholars like \cite{scott2005civilizations} who gave a detailed theory of Zomia which explains the Northern mountains of India. However, limited study has been done to actually study this gap quantitatively. Studying this gap using CS tools and statistics will give us a definite answer and either prove or disprove the Zomia hypothesis. We also looked at how Duverger's law operates within different parts of India, particularly the northern mountain states and how these variations contribute to the broader understanding of Indian electoral politics. Here, we will use Duverger's law as a proxy to study the political structure of India. This thesis will contribute to the fundamental understanding of how identities are built from a geographical perspective. It will also help us understand the role of national politics in state and how minorities have always tried to evade this control. With the growth of secessionist sentiment in North Eastern states like Nagaland and ethnic riots in Manipur as the question of formation of identities is of increased relevance.